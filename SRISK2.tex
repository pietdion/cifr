\documentclass[authoryear]{elsarticle}
\usepackage{latexsym}
%\usepackage{rotate}
\usepackage{graphics}
\usepackage{amsmath}
\usepackage{amssymb}
\usepackage{comment}
\bibliographystyle{chicago}



\newcommand{\logit}{\mathrm{logit}}
\newcommand{\I}{\mathrm{I}}
\newcommand{\E}{\mathrm{E}}
\newcommand{\p}{\mathrm{P}}
\newcommand{\e}{\mathrm{e}}
\newcommand{\vecm}{\mathrm{vec}}
\newcommand{\kp}{\otimes}
\newcommand{\diag}{\mathrm{diag}}
\newcommand{\cov}{\mathrm{cov}}
\newcommand{\eps}{\epsilon}
\newcommand{\ep}{\varepsilon}
\newcommand{\obdots}{\ddots}    % change this later
\newcommand{\Ex}{{\cal E}}
\newcommand{\rat}{{\frac{c_{ij}}{c_{i,j-1}}}}
\newcommand{\rmu}{m}
\newcommand{\rsig}{\nu}
\newcommand{\fd}{\mu}
\newcommand{\tr}{\mathrm{tr}}
\newcommand{\cor}{\mathrm{cor}}
\newcommand{\bx}[1]{\ensuremath{\overline{#1}|}}
\newcommand{\an}[1]{\ensuremath{a_{\bx{#1}}}}

\newcommand{\bi}{\begin{itemize}}
\newcommand{\ei}{\end{itemize}}

\renewcommand{\i}{\item}
\newcommand{\sr}{\ensuremath{\mathrm{SRISK}}}
\newcommand{\cs}{\ensuremath{\mathrm{CS}}}
\newcommand{\cri}{\ensuremath{\mathrm{Crisis}}}
\newcommand{\var}{\ensuremath{\mathrm{VaR}}}
\newcommand{\covar}{\ensuremath{\mathrm{CoVaR}}}
\newcommand{\med}{\ensuremath{\mathrm{m}}}
\newcommand{\de}{\mathrm{d}}
\renewcommand{\v}{\ensuremath{\mathrm{v}_q}}
\newcommand{\m}{\ensuremath{\mathrm{m}}}
\newcommand{\tvar}{\ensuremath{\mathrm{TVaR}}}



\newcommand{\eref}[1]{(\ref{#1})}
\newcommand{\fref}[1]{Figure \ref{#1}}
\newcommand{\sref}[1]{\S\ref{#1}}
\newcommand{\tref}[1]{Table \ref{#1}}
\newcommand{\aref}[1]{Appendix \ref{#1}}




\newcommand{\cq}{\ , \qquad}
\renewcommand{\P}{\mathrm{P}}
\newcommand{\Q}{\mathrm{Q}}

\newcommand{\Vx}{{\cal V}}
\newcommand{\be}[1]{\begin{equation}\label{#1}}
\newcommand{\ee}{\end{equation}}




\begin{document}

% Title of paper
\title{Systemic risk and contagion effects in Australian financial institutions and sectors}
% List of authors, with corresponding author marked by asterisk
\author{Piet de Jong,  Geoff Loudon and Weihao Choo \\[4pt]
% Author addresses
\textit{Department of Applied Finance and Actuarial Studies\\ Macquarie University, Sydney, NSW 2109.}
\\[2pt]
%E-mail address for correspondence
{piet.dejong@mq.edu.au}}

% Running headers of paper:
\markboth%
% First field is the short list of authors
{De Jong}
% Second field is the short title of the paper
{Systemic risk}

\maketitle



\section{Simplified notation}

Suppose the system is currently at time $t$ and the capital shortfall for firm $i$ at time $t+h$ is
$$
c_i \equiv  kd_i -(1-k)w_i(1+r_i)
$$
where $k$ is the prudential requirement, $d_i$ and $w_i$ are debt and equity at time $t$, and $r_i$ is the return on equity from time $t$ to $t+h$. The only random quantity in the above definition is $r_i$. The total capital shortfall in the system is
$$
c_m = \sum_i c_i = k\sum_i d_i -(1-k)\sum_i w_i(1+r_i) = k d_m - (1-k)w_m(1+r_m) 
$$
where
$$
d_m \equiv \sum_i d_i \cq w_i \equiv \sum_i w_i \cq r_m \equiv \frac{\sum_i w_ir_i}{\sum_i w_i}
$$
are the overall debt and equity for the market at time $t$ and the market return on equity from time $t$ to $t+h$. Note it is assumed that capital surpluses in certain firms are used to offset shortfalls in other firms, that is a diversification benefit is allowed in the system.

The systemic risk contribution by firm $i$ at time $t+h$ is the expected shortfall in systemic event $r_m<a$
$$
\mathrm{SRISK}_i = \E(c_i|r_m<a) = kd_i -(1-k)w_i\{1+\E(r_i|r_m<a)\} \;,
$$
and is computed from the joint probability distribution of $(r_i,r_m)$. Replace the systemic event with one having a probabilistic interpretation: $u_m < a$ where $0<a<1$ and $u_m$ is the percentile rank of $r_m$. The refined systemic event $u < a$ captures the worst $a$ of possible market returns and has probability $a$. In addition the threshold $a$ is simply the percentile rank of the original threshold. Hence the revised systemic risk of firm $i$ is
$$
\mathrm{SRISK}_i \equiv \E(c_i|u_m<a) = kd_i -(1-k)w_i\{1+\E(r_i|u_m<a)\} \;.
$$
In addition the aggregate systemic risk is
$$
\mathrm{SRISK}_m \equiv \sum_i \mathrm{SRISK}_i = \sum_i \E(c_i|u_m<a) = \E(c_m|u_m<a)
$$
$$
=kd_m -(1-k)w_m\{1+\E(r_m|u_m<a)\} \;.
$$
Note the original definition of aggregate systemic risk is the sum of positive systemic risks of individual firms
$$
\sum_i \left(\mathrm{SRISK}_i\right)_+
$$
and does not result in the above simplification. The original definition assumes no diversification: capital surpluses cannot cover shortfalls. The systemic risk of firm $i$ is positive, indicating capital shortfall, if and only if
$$
\E(r_i|u_m<a) < \frac{kd_i}{(1-k)w_i} - 1
$$




\section{Alternative measures of SRISK}

Measuring systemic risk at a single systemic event $u_m<a$ presents difficulties as to an appropriate choice of $a$. This difficulty is overcome by considering a range of systemic events with different aversion weights attached. Taking a weighted average of conditional tail expectations is akin to the construction of spectral risk measures, and results are weighted premiums or risks already discussed extensively in the literature.

Note a weighted average of the conditional tail expected return of firm $i$ is
$$
\int_0^1 \E(r_i|u_m<a) w(a) \de a = \int_0^1 \frac{w(a)}{a} \left\{ \int_0^a \E(r_i|u_m=v) \de v \right\}  \de a
$$
$$
=\int_0^1 \E(r_i|u_m=v) \int_v^1  \frac{w(a)}{a}     \de a \de v = \E\left\{\E(r_i|u_m)\phi(u_m)\right\}=\E\{r_i\phi(u_m)\}
$$
where
$$
\phi(v)\equiv \int_v^1 \frac{w(c)}{c}     \de c  
\cq \int_0^1 \phi(v) \de v = \int_0^1 \int_v^1 \frac{w(c)}{c} \de c \de v = \int_0^1 \frac{w(c)}{c} \int_0^c \de v \de c = 1\;.
$$
Hence the result is an aversion expected return of firm $i$ with aversion weights $\phi(u_m)$ or $\E\{\phi(u_m)|r\}$ integrating to one provided the same applies to $w(a)$. Note $\phi$ is a decreasing function which is straightforward from its definition. In addition the aversion adjusted expectation of $r_i$ can be written as
$$
\E\{r_i\phi(u_m)\} = \E(r_i)+\cov\{r_i,\phi(u_m)\} \;,
$$
the original expectation plus a negative risk adjustment.

Applying the same weighted averaging to SRISK yields
$$
\widetilde{\mathrm{SRISK}}_i \equiv \int_0^1 \mathrm{SRISK}_i(a) w(a) \de a = \E\{c_i\phi(u_m)\} 
$$
$$
= kd_i-(1-k)w_i[1+\E\{r_i\phi(u_m)\}] \;,
$$
which is the aversion adjusted expected capital shortfall.



\section{Standardised SRISK and layer dependence}

Standardising SRISK by subtracting the unconditional expected capital shortfall and dividing by the same assuming maximum systemic risk yields a quantity which is independent of initial debt and equity. In addition standardised SRISK reflects the local dependence between an individual firm and the market at the selected threshold of the systemic event. Hence varying the threshold yields the dependence structure of individual firm and market returns from benign events all the way to tail events.

The unconditional expected capital shortfall for firm $i$ is $\E(c_i)=kd_i-(1-k)w_i\{1+\E(r_i)\}$, suggesting the standardised SRISK
$$
\mathrm{SRISK}_i^* \equiv \frac{\mathrm{SRISK}_i-\E(c_i)}{\max(\mathrm{SRISK}_i)-\E(c_i)}
=\frac{\E(r_i)-\E(r_i|u_m<a)}{\E(r_i)-\E(r_i|u_i<a)}
$$
$$
=\frac{\cov\{r_i,\overline{I}_a(u_m)\}}{\cov\{r_i,\overline{I}_a(u_i)\}}
=\frac{\cov\{r_i,I_a(u_m)\}}{\cov\{r_i,I_a(u_i)\}} 
$$
where $\overline{I}_a(z)\equiv (z<a)$ and $I_a(z)\equiv (z>a)$ are indicators. Maximum SRISK occurs if firm return is comonotonic with the market return or $u_i=u_m$, $u_i$ being the percentile rank of $r_i$. Standardised SRISK is hence layer dependence on the original scale, and mainly lies between 0 and 1 indicating independence and comonotonicity, respectively. Varying $a$ reveals the local dependence between firm and market returns at different thresholds, with $a\rightarrow 0$ yielding the lower extreme tail dependence in a market crash of maximum severity.

Performing the same standardising on $\widetilde{\mathrm{SRISK}}_i$ which is a weighted average of SRISK for firm $i$ across various thresholds yields
$$
\widetilde{\mathrm{SRISK}}_i^* = \frac{\cov\{r_i,\phi(u_m)\}}{\cov\{r_i,\phi(u_i)\}}   \;.
$$

\section{Standardised SRISK, correlation and beta}

Taking a weighted average of the layer dependence curve yields correlation. This implies a weighted average of standardised SRISKs for a firm across different thresholds is the overall correlation between firm and market returns. This correlation is related to the beta or systematic risk of the firm in the capital asset pricing model. Hence systemic risk relates specifically to a market crash or systemic event, whilst systematic risk relates to the average over all market return possibilities.

Since 
$$
\int_0^1 I_a(u_m) \de F^-_m(a)  = \int_0^{u_m} \de F^-_m(a) = r_m-\inf(r_m) \;,
$$
taking a weighted average of standardised SRISKs for firm $i$ yields
$$
\frac{\int_0^1 \mathrm{SRISK}_i(a) \cov\{r_i,I_a(u_i)\} \de F^-_m(a) }
{\int_0^1\cov\{r_i,I_a(u_i)\} \de F^-_m(a) }
=\frac{\int_0^1 \cov\{r_i,I_a(u_m)\} \de F^-_m(a) }
{\int_0^1\cov\{r_i,I_a(u_i)\} \de F^-_m(a) }
$$
$$
=\frac{\cov\{r_i,r_m-\inf(r_m)\}}{\cov\{r_i, r_m^*-\inf(r_m)\}}
=\frac{\cov(r_i, r_m)}{\cov(r_i, r_m^*)}
=\frac{\cor(r_i, r_m)}{\cor(r_i, r_m^*)} = \frac{\beta_i\sigma_m }{\sigma_i\cor(r_i, r_m^*)}
$$
where $r_m^*$ has the same marginal distribution as $r_m$ but is comonotonic with $r_i$ and
$$
\beta_i \equiv \frac{\cov(r_i,r_m)}{\cov(r_m,r_m)} = \cor(r_i,r_m)\frac{\sigma_i}{\sigma_m} 
$$
is the beta of firm $i$ under the capital asset pricing model, where $\sigma_i$ and $\sigma_m$ are volatilities of $r_i$ and $r_m$ respectively. Hence the above weighted average of SRISKs for firm $i$ is the scaled correlation between firm and market returns, and is directly proportional to the beta of the firm.





\section{Relative systemic risk contributions}

The contribution of firm $i$ to total systemic risk is originally defined as the positive expected capital shortfall relative to the same computed over all firms in the market:
$$
\frac{ (\mathrm{SRISK}_i)_+}{\sum_i(\mathrm{SRISK}_i)_+}  \;.
$$
A large proportion indicates a systemically important firm. 

The following improves the computation in two ways. Firstly firms with capital surplus are included, since they, especially those with large surpluses, are systemically important but in a positive way by dampening the impact from distressed firms. Secondly it is more appropriate to consider the departure from the unconditional expectation to differentiate firms which only experience capital shortfall during a market downturn as opposed to those who already have large shortfalls in average conditions. The revised contribution of firm $i$ to total systemic risk is hence
$$
\frac{\mathrm{SRISK}_i-\E(c_i)}{\mathrm{SRISK}_m-\E(c_m)}
=\frac{w_i }{w_m } \frac{\E(r_i)-\E(r_i|u_m<a)}{\E(r_m)-\E(r_m|u_m<a)} \;,
$$
the product of the equity of firm $i$ relative to total market equity and return movement of firm $i$ in a market downturn relative to overall market movement.

\section{Comparing SRISK for different firms (not sure)}

Suppose of interest is whether SRISK varies similarly or different across firms as the threshold $a$ varies. High correlation implies high systemic risk since firms are simultaneously affected in a market downturn. The correlation is
$$
\cor_a\left( \mathrm{SRISK}_i(a), \mathrm{SRISK}_2(a)\right) = \cor_a \left\{\E(r_1|u_m < a),\E(r_2|u_m < a)   \right\}
$$
where the correlation is computed using a uniform distribution for $c$. 


\section{Alternative approach to systemic risk}

If $d_t$ and $w_t$ are the debt and equity of a firm at time $t$ then the capital shortfall at time $t$ is 
\be{shortfall}
kd_t  - (1-k) w_t = kd_t\left\{1-\e^{-(\ell_t+k^*)}\right\}\ ,
\ee
where 
$$
\ell_t \equiv \ln\frac{d_t}{w_t} \cq k^*\equiv\ln \frac{k}{1-k}\ .
$$
The quantity $\ell_t$ is called the log--leverage and $\ell_t>-k^*$ implies capital shortfall \eref{shortfall} is positive. 

Basel II assumes $k=0.08$ implying $k^*=-2.44$  and there is  positive  ``Basel shortfall"  if
$\ell_t > 2.44$
As the log leverage $\ell_t$  increases to 2.44 the firm approaches the Basel II threshold. 
The actual Basel II shortfall is
$$
 0.08d_t(1-\e^{-\ell^*_t})\cq \ell_t^*\equiv\ell_t+k^*=\ell_t-2.44 \ .
$$
Call $\ell_t^*$ the Basel leverage.

\subsection{Future returns}

Consider  time  $t+h$ where $h>0$.  Then
$$
\ell_{t+h} = \ln \frac{d_{t+h}}{w_{t+h}} = \ell_t -\nu_{t}\cq \nu_t\equiv \ln\frac{w_{t+h}}{w_t}  \ .
$$
where it is assumed $d_{t+h}=d_t$.  Hence $\nu_{t}$ is the return on equity over $(t,t+h)$. 
In terms of the Basel leverage $\ell_t^*$ at time $t$, the  Basel II shortfall  at time $t+h$ is,
\be{bs}
kd_t(1-\e^{-\ell^*_t+\nu_{t}})\cq k\equiv 0.08\ , 
\ee 
The Basel II limit is reached at $t+h$ if $\nu_{t}=\ell_t^*$ and  breached  if
$
\nu_{t}< \ell_t^*
$.
A breach is  unlikely if $\ell^*_t$ is low.


Engle defines the systemic risk (SRISK) of a firm as the expected Basel II shortfall at $t+h$,
$$
k d_t\E(1-\e^{-\ell^*_t+\nu_t}) =  kw_t\{1-\e^{-\ell^*_t}\E(\e^{\nu_t})\}\ ,
$$
where the expectation $\E$ is a ``stressed" expectation i.e. an expectation conditional on a stress in the system such as a major market downturn.




\subsection{Engle systemic risk}
If there are a number of  firms $i$ then $\ell_t$ and $\nu_t$ are  vectors with components $\ell_{it}$ and $\nu_{it}$, respectively, and in vector terms, $\ell_{t+h}=\ell_t - \nu_t$.   The Engle systemic risk (SRISK) of  firm $i$ as 
\be{esrisk}
 kd_{it}\{1-\e^{-\ell^*_{it}}\E(\e^{\nu_{it}})\}\ ,
\ee
where the expectation is ``stressed" meaning it is conditional on a systemic event.  Further the SRISK of the system is
\be{essrisk}
kd_{+t}\Ex\{|1-\e^{-\ell^*_{it}}\E(\e^{\nu_{it}})|^+\}\cq |x|^+\equiv \max(0,x) \ ,
\ee
where $\Ex$ denotes debt weighted averaging.

The following comments appear important:
\begin{enumerate}
\item The SRISK contribution of a firm is zero whenever
\be{cond}
\E(\e^{\nu_{it}})\le\e^{\ell^*_{it}}\ .
\ee
This inequality generally holds even if the expectation is severely stressed. 
\item  If $\nu_{it}$ is normal then 
$$
\E(\e^{\nu_{it}}) = \e^{h\mu_{it} +h^2\sigma^2_{it}/2}
$$
where the mean and standard deviation are the stressed versions.  In terms of this notation \eref{cond}
is equivalent to
$$
\mu_{it} +h\frac{\sigma^2_{it}}{2}\le \ell_{it}^* \ ,
$$
indicating a Basel breach is unlikely unless $\ell_{it}$ is near 2.44, the Basel limit.

\item The system SRISK is zero if all firm the SRISK contributions are zero that is if \eref{cond} holds for every $i$.  This would be usual in the Australian case.
\end{enumerate}

\subsection{Improved measurement of systemic risk}

Firm $i$ breaches the Basel II capital standard if, from \eref{bs},
$$
1-\e^{-\ell^*_{it}+\nu_{it}} >0 \cq \nu_{it}< \ell_{it}^*\equiv \ell_{it}-2.44  \ .
$$
For each firm $i$ define the  Basel put and standardized systemic risk as 
\be{put}
p_{it}\equiv  |1-\e^{-\ell^*_{it}+\nu_{it}}|^+\cq \E(p_{it})>0\ ,
\ee
where, as before, $\E$ denotes stressed expectation.  The systemic risk in firm $i$ depends only on the present leverage  and the lower tail of the stressed $\nu_{it}$ distribution.

The total Basel put value and hence systemic risk for firm  $i$ is   $kd_{it}\E(p_{it})$ with $k=0.08$.   This is the value of insuring firm $i$ does not breach the Basel standard.  In practice it may be appropriate to discount $p_{it}$ in \eref{put} by a risk free rate applicable over the period $(t,t+h)$.        

The put $p_{it}$ can be set up as a tradable financial contract as it relies only on the present leverage and the future return on $w_{it}$.  Market participants can value the put in  the same way as any other put and use the contract to diversify risk.  Firms can use the puts to share Basel default risk.    From the regulators perspective  the put is valued using a stressed, rather than risk neutral, expectation and regulators can monitor put values for systemic risk. Governments can charge firm $i$ the put value if there are  implicit government bailout guarantees.

In terms of the put in \eref{put}, the standardised system systemic is defined as
\be{nsrisk}
\Ex\{\E(p_{it})\} = \E\{\Ex(p_{it})\}\ , 
\ee
where $\Ex$ is an debt weighted average over firms $i$.  Scaling the standardised systemic risks $\E(p_{it})$ and $\Ex\{\E(p_{it})\}$ by $kd_{it}$ and $kd_{+t}$, respectively, yields the monetary values of  systemic risk.



  The contribution of firm $i$ to \eref{nsrisk} is  
$$
\frac{\de \Ex\{\E(p_{it})\} }{\de \E(p_{it}) } = \frac{d_{it}\E(p_{it})}{d_{+t}}\ ,
$$
which is the total value of the firm $i$ put  as a proportion of all debt.  Systemically important firms are those that heavily contribute  to the total system debt, and have high put values.  

The percentage change in the system systemic risk with respect to a one percent change in the systemic risk of firm $i$ is
$$
\frac{\de \ln \Ex\{\E(p_{it})\} }{\de \ln \E(p_{it}) } = \frac{d_{it}}{d_{+t}\Ex\{\E(p_{it})\}}\ ,
$$
which is $k$ times  the  debt in firm $i$ as a proportion of the value of the Basel system put.

Default put options similar to \eref{put} have been discussed in the insurance literature as critical to an evaluation of a firm:  see for
example \citet{merton1977analytic}, \citet{doherty1986price}, \citet{cummins1988risk}, \citet{myers2001capital} and \citet{sherris2006solvency}.

\subsection{Stressed expectations}

In the above development $\E$  denotes  stressed expectation, that is expectation conditional on a systemic event.   The systemic event is often defined in terms a market event such as a severe market downturn i.e. a very negative value for  $\nu_{mt}$.   Systemic events can be defined with respect any events.   However the discussion below uses the forward market return.  Thus the aim is to compute $\E(p_{it})$ given stress in $\nu_{it}$ defined as an extreme event couched in term of the percentile $u_{it}$ of $\nu_{it}$.

Most practical stress expectations $\E$ satisfy
\be{formula}
\frac{\E(p_{it})-\mu_{it}}{\sigma_{it}} =\sigma_\phi c_{it}\ ,
\ee
where $\mu_{it}$ and $\sigma_{it}$ are the unstressed mean and unstressed standard deviation of $p_{it}$ and $c_{it}$ is the correlation between $p_{it}$ and $\phi(u_{mt})$ where $\phi$ is a probability density on $[0,1]$ and  $\sigma_\phi$ is the standard deviation of $\phi(u)$ if  $u$ is uniform.     

Thus stressing the expectation amounts to choosing a density $\phi$ and, in effect shifting the z--scores of the $p_{it}$  
by  $\sigma_\phi c_{it}$.  The constant $\sigma_\phi$ is analytically determined from $\phi$, while the correlation $c_{it}$  is  readily estimated using simulation.  

 The proportion of system systemic risk attributable to firm $i$ is
$$
\frac{\mu_{it} +\sigma_\phi\sigma_{it}c_{it}}{\Ex(\mu_{it}) +\sigma_\phi\sigma_{\Ex(p_{it})}c_{\Ex(p_{it}),t}}\ .
$$ 
The setup is illustrated with two examples which generalise the approach of Engle.

\subsubsection{Extreme Value--at--Risk event in market return} 

If  $\phi(u)=20$ for $u\le 0.05$ and 0 otherwise then the stress event is a market return in the bottom 5\% tail.   In this case 
$$
\sigma_\phi = \cq \hat c_{it} =\frac{1}{S} \sum_s \left( \frac{p_{it}^s - \hat\mu_{it}}{\hat \sigma_{it}}\right)\left\{\frac{I(u^s_{i,t+m}<0.05)-0.05}{xxx}\right\}
$$$$
=\frac{1}{Sxxx\hat \sigma_{it}} \sum_s \left(p_{it}^s - \hat\mu_{it}\right)I(u^s_{i,t+m}<0.05)\ ,
$$$$
=\frac{1}{(xxx)\hat \sigma_{it}}\left\{\left(\frac{1}{S} \sum_{u^s_{i,t+m}<0.05} p_{it}^s \right)- 0.05\hat\mu_{it}\right\}\ ,
$$

where $s=1,\ldots,S$ denotes simulations and $I$ the indicator function. 

\subsubsection{Expected worst market return in 10 years.} 
 A more general example is $\phi(u)=n(1-u)^n$.  Then the stressed expectation is conditional the market return is the worst .... (Weihao-- can you fix this)  less than the expected minimum in $n$ independent identical situations.   In this situation simulation is problematic and computing $c_{it}$ via simulation  and using in  \eref{formula} is a viable method of estimating the stressed expectation.

 

\subsection{Contagion effects}
Definition \eref{nsrisk} can be generalized to capture contagion effects.  Consider 
\be{cont}
\Ex^j\{\E^j(p_{it})\}=\E^j\{\Ex^j(p_{it})\}\ ,
\ee
where  $\E^j$ denotes the stressed conditional expectation, conditioning on a firm $j$ breach, and $\Ex^j$ is an debt weighted average excluding firm $j$.  Firm $j$ is systemically important  if a Basel breach in firm $j$ leads to large increase in system systemic risk
$$
c_{ij}\equiv \frac{\E^j\{\Ex^j(p_{it})\}}{\E\{\Ex(p_{it})\}}-1\ .
$$

\subsection{Diversification effects} 

Suppose debt and equity are  aggregated across firms to yield $d_{+t}$ and $w_{+t}$.  The system log leverage is  
$$
\ell_{+t} =  \ln \frac{d_{+t}}{w_{+t}} =  \ln \sum_i\frac{w_{it}}{w_{+t}}\frac{ d_{it}}{ w_{it}} = \ln \Ex_w(\e^{\ell_{it}}) \ ,
$$
where $\Ex_w$ denotes an equity weighted average.  A system wide breach of the Basel II limit occurs at $t+h$ if
$$
\nu_{+t} < \ell_{+t}^*\equiv \ell_{+t} - 2.44 \ ,
$$
where $\nu_{+t}$ is the rate of return on total equity $w_{+t}$:
\be{ew}
 \nu_{+t} = \ln \frac{w_{+t+h}}{w_{+t}} = \ln \Ex_w\left(\frac{w_{i,t+h}}{w_{it}}\right) = \ln \Ex_w(\e^{\nu_{it}})\ . 
\ee

Similar to before define a put for the system and its (stressed) expected value
$$
p_{+t}\equiv |1-\e^{-\ell_{+t}^*+\nu_{+t}}|^+\cq \E(p_{+t})\le\Ex\{\E(p_{it})\}\ .
$$
The inequality occurs as a result  of diversification effects:   low liquidity  in one firm is implicitly passed on to high liquidity  firms.

\subsection{Risk weighted assets}

Total assets of a company are $d_{t}+w_{t}$.   This is substituted by risk weighted assets denoted $d_t+\e^{-v_t}w_t$ where $v_t>0$ is the implicit discounting on equity used to arrive at risk weighted assets.  
  Using risk weighted assets implies the modified leverage
$$
 \ln\left\{ \frac{d_t}{\e^{-v_t}w_t}\right\}=\ell_t +v_t\ .
$$
In terms of this setup the systemic risk in firm $i$ is then $\E(p_{it})$ computed with $\ell_t^*=\ell_t + v_t-2.44$.
$$
p_{it}\equiv  |1-(11.5)\e^{-\ell_{it}-v_{it}+\nu_{it}}|^+
$$
Since $v_{it}\ge 0$ the put is more valuable.   Note 
$$
v_t =-\ln\left\{ \frac{\Ex_\pi(a_{it})}{w_t} - \e^{\ell_t}\right\}
$$

\section*{References}
\bibliography{/users/pietdejong/documents/research/cifr/piet2}

\end{document}
