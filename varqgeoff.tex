\documentclass[authoryear]{elsarticle}
\usepackage{latexsym}
\usepackage{rotate}
\usepackage{graphics}
\usepackage{amsmath}

%\usepackage{../texstuff/rotating}

\bibliographystyle{chicago}

% xxxxxxxx


\newcommand{\logit}{\mathrm{logit}}
\newcommand{\I}{\mathrm{I}}
\newcommand{\E}{\mathrm{E}}
\newcommand{\p}{\mathrm{P}}
\newcommand{\e}{\mathrm{e}}
\newcommand{\vecm}{\mathrm{vec}}
\newcommand{\kp}{\otimes}
\newcommand{\diag}{\mathrm{diag}}
\newcommand{\cov}{\mathrm{cov}}
\newcommand{\eps}{\epsilon}
\newcommand{\ep}{\varepsilon}
\newcommand{\obdots}{\ddots}    % change this later
\newcommand{\Ex}{{\cal E}}
\newcommand{\rat}{{\frac{c_{ij}}{c_{i,j-1}}}}
\newcommand{\rmu}{m}
\newcommand{\rsig}{\nu}
\newcommand{\fd}{\mu}
\newcommand{\tr}{\mathrm{tr}}
\newcommand{\cor}{\mathrm{cor}}
\newcommand{\bx}[1]{\ensuremath{\overline{#1}|}}
\newcommand{\an}[1]{\ensuremath{a_{\bx{#1}}}}

\newcommand{\bi}{\begin{itemize}}
\newcommand{\ei}{\end{itemize}}
\newcommand{\be}{\begin{equation}}
\newcommand{\ee}{\end{equation}}
\renewcommand{\i}{\item}
\newcommand{\sr}{\ensuremath{\mathrm{SRISK}}}
\newcommand{\cs}{\ensuremath{\mathrm{CS}}}
\newcommand{\cri}{\ensuremath{\mathrm{Crisis}}}
\newcommand{\var}{\ensuremath{\mathrm{VaR}}}
\newcommand{\covar}{\ensuremath{\mathrm{CoVaR}}}
\newcommand{\med}{\ensuremath{\mathrm{m}}}
\newcommand{\de}{\mathrm{d}}
\renewcommand{\v}{\ensuremath{\mathrm{v}}}
\newcommand{\m}{\ensuremath{\mathrm{m}}}
\newcommand{\tvar}{\ensuremath{\mathrm{TVaR}}}



\newcommand{\eref}[1]{(\ref{#1})}
\newcommand{\fref}[1]{Figure \ref{#1}}
\newcommand{\sref}[1]{\S\ref{#1}}
\newcommand{\tref}[1]{Table \ref{#1}}
\newcommand{\aref}[1]{Appendix \ref{#1}}




\newcommand{\cq}{\ , \qquad}
\renewcommand{\P}{\mathrm{P}}
\newcommand{\Q}{\mathrm{Q}}





\begin{document}

% Title of paper
\title{Quantifying systemic risk and contagion effects on risk margins for interrelated financial institutions and sectors}
% List of authors, with corresponding author marked by asterisk
\author{Piet de Jong,  Geoff Loudon and Weihao Choo \\[4pt]
% Author addresses
\textit{Department of Applied Finance and Actuarial Studies\\ Macquarie University, Sydney, NSW 2109.}
\\[2pt]
%E-mail address for correspondence
{piet.dejong@mq.edu.au}}

% Running headers of paper:
\markboth%
% First field is the short list of authors
{De Jong}
% Second field is the short title of the paper
{Systemic risk}

\section*{Section A: Project Overview	}
\bi
\i Start Date:   1 May 2014			
\i Completion Date:  30 April 2014
\ei

\section{Project Name}

Systemic risk and contagion effects on risk margins for interrelated financial institutions and sectors

\section{Project Summary}

 The proposed project deals with measuring and quantifying systemic risk, contagion effects and exposure in risk margins for interrelated financial institutions and sectors and linking the same to external economic shocks.  The proposal addresses the issue of assessing the  effectiveness of stress testing in group structures as a means of enhancing the resilience of the group, and entities within the group, to financial and economic shocks and limiting intra-group contagion.
The project and methodological developments have implications of how  APRA and industry can enhance their stress testing procedures and learn more from the results of this testing.

\section{Project Team}

\begin{table}[htdp]
\begin{center}
\begin{tabular}{|l|l|l|l|l|l|}
\hline
 First & Surname & Position& Institution & Team & Email \\
Name          &                &               &   & Position                   &    \\

\hline
 Piet &  De Jong & Professor of & FBE& Leader& piet.dejong\\
         &                &  Actuarial                 & Macquarie&                    &@mq.edu.au\    \\
         &                &  Studies                                           & University &                    &            \\
 \hline
 Geoff & Loudon & Associate & FBE& Investigator& geoff.loudon\\
         &                &  Professor                & Macquarie &                    &@mq.edu.au\    \\
                  &                &                                             & University &                    &            \\
 \hline
  Weihao &  Choo& Senior  & MSIG  &Investigator&chooweihao\\
           &                & Actuarial                                            & Holdings &                    &            \\
           &                & Executive               & (Asia) &                    &  @gmail.com  \\
  \hline
\end{tabular}
\end{center}
\label{default}
\end{table}%

\newpage

\section{Budget (cash request only): }
\begin{table}[htdp]
\begin{center}
\begin{tabular}{|c|c|}
\hline
\multicolumn{2}{|c|}{Year 1  (Total)} \\
\hline
CIFR & Member \\
\hline
\$100,000 & -- \\
\hline
\end{tabular}
\end{center}
\label{default}
\end{table}%



\section{Relevant CIFR Key Area of Interest }
\bi
\i  Systemic risk  
\ei
Assessing and the level systemic risk in Australia requires a coherent  methodology to measure the same and implement them in the Australian financial system context.    A number of methodologies have recently been proposed to assess systemic risk (see references later).   A major aim of the current project is to critique,   improve upon these methodologies and apply and implement such improved technologies in the Australian context.    
 
Present regulations are based on standard approaches such as $\var_q$.   Such regulations and capital requirements are mainly ``standalone" where individual industries are analysed on a standalone basis  using similar stress scenarios.    Non--standalone approaches implicit in systemic risk calculations may lead to different margins, uncover systematic biases or tighter or amplify confidence intervals around margins in line with economic cycle.   Both over and underestimation of capital margins impose costs on the economy.    
 

\section{Research Proposal of interest to:  APRA, RB}

The current proposal  arises out of the following target areas identified by CIFR and drawn to my attention  by Professor David Gallagher:

\begin{itemize}

\item  Assess the level of systemic risk in the Australian economy and in the ADI, insurance and superannuation industries, and identify potential domestic and external shocks to the Australian economy, and potential methods to protect against or remediate these shocks.

\item  Assess the effectiveness of stress testing in group structures as a means of enhancing the resilience of the group, and entities within the group, to financial and economic shocks and limiting intra-group contagion.

\item   Consider how APRA and industry could enhance their stress testing procedures and learn more from the results of this testing.
\end{itemize}

\subsection*{Contact with other individuals} 
\bi
\i   APRA -- Charles Littrell.   On 4 March 2014 the Team Research Leader met the Mr Charles Littrell of APRA to discuss the CIFR  target research areas.   The discussion ranged over a number of topics related to the same and the attitude APRA may take to any proposals.  The discussion also canvassed the need to design appropriate quantitative methodologies and data bases for any empirical assessments.
\i   RBA -- no contact.
\ei

\section{Capability Paragraph} 

\bi 
\i Piet de Jong
\ei

Forty years' experience in the quantitative modelling uncertainty and risk including risks arising in insurance (general and mortality), time series dynamics, finance, capital allocation, health economics. 

\bi
\i Geoff Louden
\ei

Extensive experience in conducting and supervising empirical research in finance. Research area includes financial risk management and security pricing, modelling inter-relations among risks, returns and underlying factors, especially during times of market crisis. Expertise in financial econometrics including application of regime-switching models; multivariate GARCH style models; stochastic volatility estimation; etc.

\bi
\i Weihao Choo
\ei

Fully qualified actuary (FIAA) with extensive industry experience in stress testing and capital modelling, liability valuation, portfolio monitoring and pricing, investment modelling, and business planning.  Academic research expertise as evidenced in  two publications in international journals (exceptional for someone his age).  Presently finishing PhD while working in industry.   PhD research relates to risk and risk measurement and capital modelling and allocation.

\newpage

\section{ Certifications}

\subsection*{Certification by Team Leader } 
I, Piet de Jong, certify that:
\bi 
\i	All the details on this FP are true and complete; 
\i	I have notified CIFR of any actual or potential conflicts of interest I may have in relation to the FP and I undertake that, if the FP is successful, I will notify CIFR of any conflicts of interest which arise subsequent to the submission of the FP; 
\i	I will notify CIFR if there are any changes in my circumstances which may impact on my eligibility to participate in, or ability to perform, the project subsequent to the submission of this FP; 
\i	In participating in this FP, I consent to CIFR copying, disclosing and otherwise dealing with information contained in the Proposal, for the purpose of considering this proposal and making decisions as to the funding round;  
\i	All information contained in the FP is both current and accurate; 
\i	The work proposed in this project is not funded elsewhere; 
\i	The work proposed is unique and that it has not been fully or partially completed elsewhere; and
\i	All named researchers have agreed to participate and agree to an immediate CIFR announcement should funding be approved.
\ei

\begin{table}[htdp]
\begin{center}
\begin{tabular}{|l|l|l|l|}
\hline
Signature& Name & Position & Date\\
\hline
& Piet de Jong & Professor and Team Leader & 03 March 2014 \\
\hline
\end{tabular}
\end{center}
\label{default}
\end{table}%

\newpage			
\subsection*{Certification by Organisations contributing to the project} 
			
I certify that: 
\bi
\i	My organisation supports the FP and will contribute the resources outlined in the FP. 
\i	If teaching relief is requested in the FP the Organisation approves the relief. 
\ei
\begin{table}[htdp]
\begin{center}
\begin{tabular}{|l|l|l|l|}
\hline
Signature& Name & Position & Date\\
\hline
&  & &  \\
\hline
\end{tabular}
\end{center}
\label{default}
\end{table}%

\newpage

\section*{Section B: Project Objectives, Significance and Policy Implications}

Our starting point for the proposed research is the recent literature and the CIFR targeted areas and APRA aims and functions.
This recent literature includes
\cite{adrian2011covar}
\cite{acharya2012capital}
\cite{acharya2012measuring}
and \cite{brownlees2010volatility}.   The proposed research aims to extend and apply these techniques particularly in relation to the entities regulated by APRA.   Thus our  broad aim is to develop, implement and bring to bear recent developments in stress testing  on the aims of APRA in general and the CIFR targeted research areas detailed above.   

\subsection*{Improved  measures of contagion and systematic risk}
\renewcommand{\c}{\ensuremath{\mathrm{CoV_q}}}
\renewcommand{\v}{\ensuremath{\mathrm{VaR_q}}}

$\covar_q$ as proposed in \cite{adrian2011covar} suffers from a number of difficulties:
\bi
\i Couched in terms of $\var_q$ which contains the scale of the original measurements.   It is worthwhile to have a scale independent measures.

\i  Conditioning  on $\var_{0.5}$ is undesirable and relatively intractable.  In our proposed we reference stress with respect  to the unconditional $\var_q$.   This permits a more transparent analysis and estimation. 
\i  Our approach will separate out the effect of marginal distributions and interdependence and econometrically relate these these structures separately to external variables including shocks and drivers of systemic risk.
\ei

\subsection*{Significance of the project and  policy implications}

Relating marginal and joint distributions separately to external drivers allows for a more cogent and coherent stress testing including the estimation of contagion effects,  exposure effects and systemic risk across related entities and different financial sectors.  Improved stress testing, estimation of risk effects and transmission of shocks through the financial system will make for more cogent prudential policy, prudential margin setting and sources of risk to the financial system.

\subsection*{Technical background and path to improved stress testing in the context of contagion and external shocks}

Our developments will be based on the following definition and econometric implementation:
\be\label{ncovar2}
\c(x,y) \equiv \v(x|y>q) - \v(x)\cq \c(x)\equiv \c(x,x)\ .
\ee 
It is shown  that \eref{ncovar2} is a more robust and extensible definition than has been proposed in the literature  and more readily amenable and useful to empirical work. 

The  proposed  research agenda of this project begins by initially considering $u$ and $v$ uniform random variables on $[0,1]$. Then obviously  $\v(u)=\v(v)=q$.   Define $u^+\equiv\v(u|v>q)$ as the $\var_q$ of $u$ given $v$ exceeds its $\var_q$:
\be\label{Qdef}
 \P(u\leq u^+|v>q)=q \cq 0<q<1\ .
\ee
The left hand side equals
$$
 \frac{\P(u\leq u^+,v>q)}{1-q} 
=\frac{u^+-C(u^+,q)}{1-q} \ ,
$$
where  $C(u,v)$ is the joint distribution (copula) of $u$ and $v$.
Rearranging yields
\be\label{uplus}
u^+ \equiv \v(u|v>q) = q(1-q)+C(u^+,q)\ .
\ee
If $u$ and $v$ are independent then $C(u,v)=uv$ and  $u^+=q$.   If $u=v$ then 
 $u^+=q+q(1-q)=2q-q^2$ and $u^+-q=q(1-q)$.  Thus if $u$ and $v$ are non--negatively related, $0\le u^+-q\le q(1-q)$.

\bi  
\i Note
$
u^+_{t+1} = q(1-q)+C(u_t^+,q)
$.
Iterate this equation and make $C$ a function of forcing variables.  e.g. a Clayton copula where the parameter is a function of forcing variables.
\i  If $u$ and $v$ are negatively dependent then define
$$
\c(u,v)\equiv -\c(u,1-v)
$$
\ei 
 
In terms of  $u^+\equiv\v(u|v>q)$, define the contagion effect of $v$ on $u$ as
\be\label{ruv}
 \beta_{uv} \equiv \frac{u^+-q}{q(1-q)}=\frac{\c(u,v)}{\c(v)}  \ .
\ee
Thus $\beta_{uv}$ is the change in $\var_q$  of $u$ given $v$ becomes $q$--stressed as a proportion of the change if $u=v$.   For positively dependent random variables $0<\beta_{uv}\le 1$ with the lower and upper limits attained under independence and perfect dependence, respectively.    If $u$ and $v$ are negatively dependent then $1-1/(1-q)\le \beta_{uv}<0$.   Negative dependence is not be studied in great detail in this project.  Note $\beta_{uv}\ne \beta_{vu}$.

Furthermore we may define quantities such as
$
u^- \equiv \var_q(u|v\le q)
$
measuring the impact of a non distressed state in $v$.  For brevity we do not dwell on these constructs in this writeup although the ramifications and potential uses of these constructs will be  investigated in the research.

\subsection*{Contagious stress effects for financial variables and the contagion matrix}

We now discuss actual variables on actual scales.   Suppose $F_x$ and $F_y$ are the marginal distributions of $x$ and $y$ with $x=F_x^-(u)$ and $y=F_y^-(v)$.   Then the contagion effect 
 of $y$ on $x$ is defined as the change in $\var_q(x)$ when $y$ becomes $q$--distressed as in \eref{AB2}, equal to
 \be\label{covar}
\covar_q(x,y)\equiv\v\{x|y>\v(y)\} - \v(x) 
\ee
\be
= F_x^-\{q+\beta_{uv}q(1-q)\}-F_x^-(q) \approx \frac{q\beta_{uv}}{\lambda}=\frac{ \c(u,v)}{q'} \ ,
\ee
where  $'$ denotes differentiation and $\lambda$ is the hazard of $x$ at $x=\var_q(x)$.  If $F_x$ is linear then the approximation is exact.   Hence it is appropriate to scale $x$ such that $F_x^-$ is linear in the tail.   Rescaling has no effect on the copulas connecting variables.

If $x$ is vector then $\c(x)$ is the (non--symmetric) matrix with entries $\c(x_i,x_j)$ and 
$$
R\equiv\c(x) = D^{-1}\times \c(u)\cq D=\diag\left(q_1',\ldots,q_m'\right)\ ,
$$ 
where $m$ is the number of components in $x$. 

\subsection*{Econometric implementation}

The above development provides a framework for linking bivariate copulas and marginals to external variables and shocks study the impact of the same on stresses within the system and the contagious effects of crises.   Proposed econometric analysis will be as in \cite{brownlees2010volatility}.


 
 \newpage
\section*{Section C: Data, Method and Outputs}

-- Include a detailed discussion of the data needed for the project, the proposed method and the outputs (Maximum two pages).


Advice and cooperation is requested  from appropriate research personal within APRA to assist in both development and implementation. 


\section*{Timetable for the delivery of Outputs }

\begin{table}[htdp]
\begin{center}
\begin{tabular}{|l|l|l|}
\hline
Date	& Output& Details (of proposed Output)\\
\hline
&  &   \\
\hline
\end{tabular}
\end{center}
\label{default}
\end{table}%

\newpage
\section*{Section D: Research Record and References}
\subsection*{1.  Research Record} 

\bi
\i   Piet de Jong, Team Leader.   

The following publications  (in no particular order) detail my expertise in designing and, where appropriate, implementing  tools to quantify, assess and model uncertainty related to finance, demographics and economics.  (see also joint publications with Weihao Choo given under the latter's heading)

\bibitem[\protect\citeauthoryear{De~Jong}{De~Jong}{2012}]{de2012modeling}
De~Jong, P. (2012).
\newblock Modeling dependence between loss triangles.
\newblock {\em North American Actuarial Journal\/}~{\em 16\/}(1), 74--86.

\bibitem[\protect\citeauthoryear{De~Jong}{De~Jong}{2006}]{dejong2006frt}
De~Jong, P. (2006).
\newblock {Forecasting Runoff Triangles}.
\newblock {\em North American Actuarial Journal\/}~{\em 10\/}(2), 28.

\bibitem[\protect\citeauthoryear{De~Jong}{De~Jong}{1989}]{DeJong:89}
De~Jong, P. (1989).
\newblock Smoothing and interpolation with the state-space model.
\newblock {\em Journal of the American Statistical Association\/}~{\em
  84\/}(408), 1085--1088.

\bibitem[\protect\citeauthoryear{De~Jong}{De~Jong}{1991}]{DeJong:91a}
De~Jong, P. (1991).
\newblock The diffuse {K}alman filter.
\newblock {\em Annals of Statistics\/}~{\em 19\/}(2), 1073--1083.

\bibitem[\protect\citeauthoryear{De~Jong and Boyle}{De~Jong and
  Boyle}{1983}]{DeJong&Boyle:83}
De~Jong, P. and P.~Boyle (1983).
\newblock Monitoring mortality: a state-space approach.
\newblock {\em Journal of Econometrics\/}~{\em 23}, 131--146.

\bibitem[\protect\citeauthoryear{De~Jong and {C}hu-{C}hun-{L}in S.}{De~Jong and
  {C}hu-{C}hun-{L}in S.}{2003}]{DeJong&Chuchunlin:2003}
De~Jong, P. and {C}hu-{C}hun-{L}in S. (2003).
\newblock Smoothing with an unknown initial condition.
\newblock {\em Journal of Time Series Analysis\/}~{\em 24\/}(2), 141--148.

\bibitem[\protect\citeauthoryear{De~Jong and Ferris}{De~Jong and
  Ferris}{2006}]{DeJong&Ferris:2006}
De~Jong, P. and S.~Ferris (2006).
\newblock Adverse selection spirals.
\newblock {\em Astin Bulletin\/}~{\em 36\/}(2), 589--628.

\bibitem[\protect\citeauthoryear{De~Jong and Heller}{De~Jong and
  Heller}{2008}]{dejong2008glm}
De~Jong, P. and G.~Heller (2008).
\newblock {\em {Generalized Linear Models for Insurance Data}}.
\newblock Cambridge University Press.

\bibitem[\protect\citeauthoryear{De~Jong and Marshall}{De~Jong and
  Marshall}{2007}]{DeJong&Marshall:2007}
De~Jong, P. and C.~Marshall (2007).
\newblock Mortality projection based on the {W}ang transform.
\newblock {\em ASTIN Bulletin\/}~(1), 149--162.

\bibitem[\protect\citeauthoryear{De~Jong and Penzer}{De~Jong and
  Penzer}{1998}]{DeJong&Penzer:98}
De~Jong, P. and J.~R. Penzer (1998).
\newblock Diagnosing shocks in time series.
\newblock {\em Journal of the American Statistical Association\/}~{\em
  93\/}(442), 796--806.

\bibitem[\protect\citeauthoryear{De~Jong and Shephard}{De~Jong and
  Shephard}{1995}]{DeJong&Shephard:95}
De~Jong, P. and N.~Shephard (1995).
\newblock The simulation smoother for time series models.
\newblock {\em Biometrika\/}~{\em 82}, 339--350.

\bibitem[\protect\citeauthoryear{De~Jong and Zehnwirth}{De~Jong and
  Zehnwirth}{1983b}]{DeJong&Zehnwirth:83a}
De~Jong, P. and B.~Zehnwirth (1983b).
\newblock Claims reserving, state-space models and the {K}alman filter.
\newblock {\em Journal of the Institute of Actuaries\/}~{\em 110}, 157--181.


\i   Geoff Louden, Principal Researcher

Liu, J., Loudon, G., Milunovich G., Linkages between international REITs: the role of economic factors, Journal of Property Investment \& Finance, 30(5), 2012, 473�492.

Dean, W., Faff, R., Loudon, G., Asymmetry in return and volatility spillover between equity and bond markets in Australia, Pacific�Basin Finance Journal, 18(3), 2010, 272�289.

Hobbes, G., Lam, F., Loudon, G., Regime shifts in the stock�bond relation in Australia, Review of Pacific Basin Financial Markets and Policies, 10(1), 2007, 81�99.

Loudon, G., Okunev, J., White, D., Hedge fund risk factors and the Value�at�Risk of fixed income trading strategies, The Journal of Fixed Income 16(2), 2006, 46�61.

Loudon, G., Is the risk�return relation positive? Further evidence from a stochastic volatility in mean approach, Applied Financial Economics 16(13), 2006, 981�992.

Loudon, G., Financial risk exposures in the airline industry: evidence from Australia and New Zealand, Australian Journal of Management 29(2), 2004, 295�316.

Loudon, G., Watt. E., Yadav, P., An empirical analysis of alternative parametric ARCH models, Journal of Applied Econometrics 15(2), 2000, 117�136.

Loudon, G., Foreign exchange exposure and the pricing of currency risk in equity returns: Some Australian evidence, Pacific�Basin Finance Journal 1(4), 1993, 335�354.

Loudon, G., The foreign exchange operating exposure of Australian stocks, Accounting and Finance 33(1), 1993, 19�32.

Loudon, G., American put pricing: Australian evidence, Journal of Business Finance and Accounting 17(2), 1990, 297�321.

\i   Weihao Choo,  Principal Researcher


\bibitem[\protect\citeauthoryear{Choo and De~Jong}{Choo and
  De~Jong}{2009}]{choo2009loss}
Choo, W. and P.~De~Jong (2009).
\newblock {Loss reserving using loss aversion functions}.
\newblock {\em Insurance Mathematics and Economics\/}~{\em 45\/}(2), 271--277.

\bibitem[\protect\citeauthoryear{Choo and De~Jong}{Choo and
  De~Jong}{2010}]{choodejong:2010}
Choo, W. and P.~De~Jong (2010).
\newblock {Determining and Allocating Diversification Benefits for a Portfolio
  of Risks}.
\newblock {\em Astin Bulletin\/}~{\em 40\/}(1), 257--269.

\ei

\newpage
\subsection*{2.  Grants} 
\bi
\i   Piet de Jong, Team Leader

Most of my research  requires limited money -- just lots of time and concentration.   Despite this I have held a number of research grants mainly to facilitate my ongoing research program:
\bi
\i   While at the University of British Columbia from 1982--2001 (Professor of Statistics) I held  NSERC operating grants to facilitate my ongoing research program.   These are equivalent to ARC discovery grants except that the grant is based on performance rather than detailed aspirations to research one or other issue.    The individual research program and operating grant  led to many highly cited publications in A* international academic journals.  
\i  While at the University of Amsterdam (Visiting Research Fellow) in 1987 I was funded under the Dutch government's ZWO initiative.   This led to a number of world class highly cited academic publications.
\i  While at the London School of Economics (Reader in Statistics) from 1995-1998.   I  held NSERC operating grants.   Again this led to a number of highly cited A* international academic publications.
\i  While at Macquarie University 2003-2014 (Professor of Actuarial Studies)  I have held grants from Macquarie University (jointly funded with APRA) and the Institute of Actuaries.    Again this has led to my more recent A* international publications  including a jointly authored book published by Cambridge University Press.  
\ei   
\i   Geoff Louden, Principal Researcher

No recent grants

\i   Weihao Choo,  Principal Researcher

Early stage researcher.   Just now finishing PhD.


\ei

\newpage
\section*{Section E: Detailed Budget and Justifications}

\begin{table}[htdp]
\begin{center}
\begin{tabular}{|l|c|c|c|c|}
\hline
Expenditure&\multicolumn{4}{c|}{Year 1 -- Total}\\
\hline
& Cash & Cash & In Kind &  Other \\
& (CIFR) & (Member)& & \\
\hline
Staff & & & & \\ 
-- Teaching relief for De Jong & & & 47,802 & \\
for 2 months, 28\% for oncosts & & & & \\
\& 30\% infrastructure: & & & & \\
$172,363\times1.28\times2/12\times1.3$& &  & & \\
-- Teaching relief for Loudon & & & 18,720  & \\
for 1 month1, 28\% for oncosts & & & & \\
\& 30\% infrastructure: & & & & \\
$135,000\times1.28\times1/12\times1.3$& &  &  &\\
\hline
Equipment & 2,500 & 2,500 & & \\
Computer and related items & & & & \\
\hline
Data & & & & \\
\hline
Travel & 10,000 & 10,000 & & \\
2 researchers x 2 conferences & & & & \\
\hline
Fellowships & 25,000 & 25,000 & & \\
Weihao Choo & & & & \\
\hline
Scholarships & & & & \\
\hline
Research Assistant & 25,531& 25,531& 15,319& \\
Cash: RA, top of Level 6 band& & & & \\
 grossed up by on costs, 3 days pw:& & & & \\
$66487\times1.28\times0.6$& & & & \\
In kind: Infrastructure & & & & \\
 @ $30\%: 51,062.02\times 0.3$& & & & \\
\hline
Other costs & & & & \\
\hline
Total & & & & \\
\hline
\end{tabular}
\end{center}
\label{default}
\end{table}%

\subsubsection*{Basis for teaching-relief request}

De Jong requests funding for one month�s teaching relief, during the second half of 2014. The purpose is to give effective direction to the project for a period without the  burden of teaching and convening courses.

\subsubsection*{Basis for RA request}
A research assistant at the top of the Level 6 band is required for data analysis and assistance with programming. His or her tasks will be supervised by the researchers, and the appointment would be for 3 days per week for the duration of the project. The tasks are twofold. The first is to gather appropriate data and secondly to program methods in R.

\subsubsection*{Basis for Fellowship request}
The fellowship will facilitate one of the researchers (Weihao Choo) to devote appropriate time to the project.


\subsubsection*{Basis for Travel request}
Present  research at two international conference.  Note that workshopping the paper with industry practitioners and researchers at one international conference is required to maximise the impact of this research.
\newpage



 	
\section*{Bibliography}
\bibliography{/Applications/Tex/piet2}


\end{document}
