\documentclass[authoryear]{elsarticle}
\usepackage{latexsym}
%\usepackage{rotate}
\usepackage{graphics}
%\usepackage{amsmath}
%hello
\bibliographystyle{chicago}



\newcommand{\logit}{\mathrm{logit}}
\newcommand{\I}{\mathrm{I}}
\newcommand{\E}{\mathrm{E}}
\newcommand{\p}{\mathrm{P}}
\newcommand{\e}{\mathrm{e}}
\newcommand{\vecm}{\mathrm{vec}}
\newcommand{\kp}{\otimes}
\newcommand{\diag}{\mathrm{diag}}
\newcommand{\cov}{\mathrm{cov}}
\newcommand{\eps}{\epsilon}
\newcommand{\ep}{\varepsilon}
\newcommand{\obdots}{\ddots}    % change this later
\newcommand{\Ex}{{\cal E}}
\newcommand{\rat}{{\frac{c_{ij}}{c_{i,j-1}}}}
\newcommand{\rmu}{m}
\newcommand{\rsig}{\nu}
\newcommand{\fd}{\mu}
\newcommand{\tr}{\mathrm{tr}}
\newcommand{\cor}{\mathrm{cor}}
\newcommand{\bx}[1]{\ensuremath{\overline{#1}|}}
\newcommand{\an}[1]{\ensuremath{a_{\bx{#1}}}}

\newcommand{\bi}{\begin{itemize}}
\newcommand{\ei}{\end{itemize}}
\newcommand{\be}{\begin{equation}}
\newcommand{\ee}{\end{equation}}
\renewcommand{\i}{\item}
\newcommand{\sr}{\ensuremath{\mathrm{SRISK}}}
\newcommand{\cs}{\ensuremath{\mathrm{CS}}}
\newcommand{\cri}{\ensuremath{\mathrm{Crisis}}}
\newcommand{\var}{\ensuremath{\mathrm{VaR}}}
\newcommand{\covar}{\ensuremath{\mathrm{CoVaR}}}
\newcommand{\med}{\ensuremath{\mathrm{m}}}
\newcommand{\de}{\mathrm{d}}
\renewcommand{\v}{\ensuremath{\mathrm{v}_q}}
\newcommand{\m}{\ensuremath{\mathrm{m}}}
\newcommand{\tvar}{\ensuremath{\mathrm{TVaR}}}



\newcommand{\eref}[1]{(\ref{#1})}
\newcommand{\fref}[1]{Figure \ref{#1}}
\newcommand{\sref}[1]{\S\ref{#1}}
\newcommand{\tref}[1]{Table \ref{#1}}
\newcommand{\aref}[1]{Appendix \ref{#1}}




\newcommand{\cq}{\ , \qquad}
\renewcommand{\P}{\mathrm{P}}
\newcommand{\Q}{\mathrm{Q}}





\begin{document}

% Title of paper
\title{Systemic risk and contagion effects in Australian financial institutions and sectors}
% List of authors, with corresponding author marked by asterisk
\author{Piet de Jong,  Geoff Loudon and Weihao Choo \\[4pt]
% Author addresses
\textit{Department of Applied Finance and Actuarial Studies\\ Macquarie University, Sydney, NSW 2109.}
\\[2pt]
%E-mail address for correspondence
{piet.dejong@mq.edu.au}}

% Running headers of paper:
\markboth%
% First field is the short list of authors
{De Jong}
% Second field is the short title of the paper
{Systemic risk}

\maketitle

\section{Literature review}

Our starting point for the proposed research is the recent literature and the CIFR targeted areas and APRA aims and functions.
This recent literature includes the following
\cite{adrian2011covar},
\cite{acharya2012capital},
\cite{acharya2012measuring}
and \cite{brownlees2010volatility}.   The proposed research aims to extend and apply these techniques particularly in relation to the entities regulated by APRA.   Thus our  broad aim is to develop, implement and bring to bear recent developments in stress testing  on the aims of APRA and the CIFR targeted research areas detailed above.   

\section{Improved  measures of contagion and systematic risk}
\renewcommand{\c}{\ensuremath{\mathrm{CoVaR_q}}}
\renewcommand{\v}{\ensuremath{\mathrm{VaR}_q}}

$\covar_q$ as proposed in \cite{adrian2011covar} is a basis for proposed measures contagion, exposure and systemic risk.   It  suffers from a number of drawbacks:
\bi
\i Couched in terms of $\var_q$ which contains the scale of the original measurements.   It is worthwhile to have measures and techniques robust to scale.
\i  Conditioning  on $\var_{0.5}$ is undesirable and relatively intractable.  In our proposal we reference stress with respect  to the unconditional $\var_q$.   This permits a more transparent analysis and estimation. 
\i  Our proposed approach  separates out scale effects and interdependence effects and aims to  relates these separately to external variables including shocks and drivers of systemic risk.   Thus $\var_q$ movements due to scale are disentangled from movements due to codependence with separate driver responses.
\ei

\section{Significance of the project and  policy implications}

Understanding the impact of external shocks and their propagation through   the financial system is vital for managing and remediating systemic risk. Effective regulation is dependent upon the development of a robust and reliable set of appropriate risk measures.  We propose new measures of systemic risk that relate marginal and joint distributions separately to external drivers. This allows for more cogent and coherent stress testing as it includes the estimation of contagion effects, exposure effects and systemic risk across related entities and different financial sectors. Improved stress testing, estimation of risk effects and transmission of shocks through the financial system will make for more cogent prudential policy, prudential margin setting and better identify sources of risk to the financial system.

\section{Percentile sensitivity and contagion}

Improved stress testing in the context of  external shocks is based on the following definition
\be\label{ncovar2}
q_{yx} \equiv q_{y|x>q_x} - q_y\ ,
\ee 
where the $q_y$, $q_x$ and $q_{y|x>q_q}$ are the $q$--quantiles in the distributions of $y$, $x$ and the conditional distribution of $y$ given $x>q_x$, respectively.
Thus  $q_{yx}$ is the change in the \v\ of $y$ when moving from a situation where $x$ is not known, to where  $x$ is known to be stressed, $x>q_q$. 
It is shown  that \eref{ncovar2} is a more robust and extensible definition than has been proposed in the literature  and more readily amenable and useful to empirical work. 

If $v$ and $u$ are uniform random variables on $[0,1]$ then  
\be\label{Qdef}
q_u=q_v= q=\P(u\le q_*|v>q) =  \frac{\P(u\le q_*,v>q)}{1-q}  = \frac{q_*-C(q_*,q)}{1-q}
\ee
where $q_*=q_{u|v>q}$ and $C(u,v)$ is the joint distribution (copula) of $u$ and $v$.
Rearranging yields
\be\label{uplus}
q_* -C(q,q_*)= q(1-q)\cq q_{uv}=C(q+q_{uv},q)-q^2\ .
\ee
This provides an implicit equation for $q_{vu}$, solved by  root finding algorithms given the copula $C(u,v)$.

If $u$ and $v$ are independent then $C(u,v)=uv$ and  $q_{vu}=0$.   If $u=v$ then $C(u,v)=\min(u,v)$ and  $q_{uv}
=\min(q+q_{uv},q)-q^2=q(1-q)$.  If $u$ and $v$ are exchangeable, $C(u,v)=C(v,u)$, then 
$$
q_{vu} = C(q,q+q_{vu})-q^2=C(q+q_{vu},q)-q^2=q_{uv}\ .
$$
Finally  if $q'_{uv}$ is the derivative of $q_{uv}$ with respect to $q$ then
$$
q'_{uv}= (1+q'_{uv})C_u(q+q_{uv},q)+C_v(q+q_{uv},q)-2q
$$
where $C_u$ and $C_v$ are the partial derivatives of $C$ with respect to $u$ and $v$, respectively.

For a given vector of uniform variables define $Q$ as the matrix with entries 
$$
\frac{q_{uv}}{q(1-q)}
$$   
Each column of $Q$ indicates the change \v\ in each of the row variables when the column variable  is stressed, as a proportion of the change if $u=v$.  The diagonal of $Q$ is 1. 

\section{Financial sensitivity and contagion}
 
 If $F_y$ is the marginal distribution of $y$ then the change in the \v\ of $y$ when $x$ becomes stressed is
 $$
 q_{yx}= F_y^-(q+q_{vu}) - q_y \approx q_{vu}(F_y^-)'(q) = \frac{q_{vu}}{f_y(q_y)} \ .
 $$
The approximation follows from a first order Taylor expansion.    Thus the change in the quantile of $y$ when $x$ becomes stressed depends only on  $F_y(y)$ and $C(u,v)$:  the marginal $F_x(x)$ is not relevant.  Using the same argument
 $$
 q_{yy}=  F_y^-(2q-q^2) - q_y\approx \frac{q(1-q)}{f_y(q_y)}
 \qquad\Rightarrow\qquad
 \frac{q_{yx}}{q_{xx}} \approx  \frac{q_{vu}}{q(1-q)}=\frac{q_{vu}}{q_{uu}}\ .
 $$
 Thus $Q$ based on percentiles approximates the contagion matrix of the original variables.
(Say something about when the approx is good/bad and the scaling on $y$ that may make the approximation better.)

A column of $Q$ contains the \v\ sensitivities of each variable when the  corresponding column variable is stressed.   Each row of $Q$ displays the \v\ sensitivity of the variable corresponding to the row to all the variables.  Define
$$
Q_* \equiv \frac{1}{p-1}(Q-I)=UDV'\cq s\equiv Q_*1\cq c\equiv Q_*'1\ .
$$$$
Q=I + s1'+ UDV'\cq s\equiv \frac{1}{p-1}(Q-I)1
$$
Then $s$ is the vector of average \v\ sensitivity of each variable to all others.  Further $c$ is the average \v\ impact of each variable on all others.   The matrices $U$, $D$ and $V$ define the singular value (svd) decomposition of $Q_*$, arranged so that diagonal matrix $D$ has the singular values on  the diagonal in descending order.   If $b=d_1u_1$ and $c=v_1$ where $u_1$, $d_1$ and $v_1$ are the first column, top diagonal entry, and first column of $U$, $D$ and $V$ respectively, then  for two variables $y\ne x$ in $Q$,
$$
q_{yx} = s_y+ b_yc_x +\eps_{yx}
$$
This states that percentage sensitivities are, apart from the ``error" $\eps_{yx}$, an average sensitivity plus a scaled response to the contagious effect of the $x$ variable.   The contagious effects contained in $c$ are estimated by maximising the explanation of $Q$.


The vector $1'Q$ sums  the changes in \v\ when each of the column variables is stressed, and writes this as a proportion of the change in the variable being stressed.   These proportional sums, subtracting 1 and divided by  $p-1$ where $p$ is the number of variables, measures the average contagion of each variable on all others:  $c'=(p-1)^{-1}1'(Q-I)$ or $c=(p-1)^{-1}(Q-I)'1$

Alternatively the vector $Q1$ sums the changes in \v\ of each row variable when all the column variables are stressed.   Again it is appropriate to remove the effect of a variable on itself and consider the average over the remaining variables:  $s\equiv(p-1)^{-1}(Q-I)1$.  If $Q_*\equiv (p-1)^{-1}(Q-I)$ then $s=Q_*1$ and $c=Q'_*1$ are the vectors of sensitivities and contagions, respectively.  If $Q=I$ then $s=c=0$.   If $Q=11'$ then $s=c=1$.


 
 \section{Systemic risk and causal chains}
A rank one approximation to the matrix  $Q$ is 
 $
 Q  \approx  sc' 
 $.
 Vector $c$ is an index of the contagious  impact of each of variables on the others while $s$ measures the sensitivity of each variable to each of the others.  The vectors $s$ and $c$ are derived from the singular value decomposition $Q=UDV'$ where $U'U=V'V=1$  and  where $s$ and $c$ are the first column of $UD$ and $V$  respectively, assuming the svd is organised so that the singular values in the diagonal matrix $D$ are organised from largest to smallest.   The appropriateness of the summarisation $sc'$ is measured with $\tr(Q-sc')$.
 
If the variables are independent then $Q=I$ and both $s$ and $c$ equal a column of the identity matrix with $sc'$ a matrix of 0's except in a single diagonal position where it is 1.   Then all but one variable has a contagion effect and only that variable is sensitive to the contagion provided by the  variable.

If the variables are comonotonic then $Q=11'$ and $s=1$ and $c=1$ where 1 denotes a vector of ones.   Thus the rank 1 approximation is exact and each variable is equally contagious and equally sensitive.
 
 Note that 
 $$
 Q^{n} \approx (s'c)^{n-1}sc'=\ ,
 $$
 
 
 If the random variables are independent the $Q-I=0$ and $a=b=k=0$ and there is no error in the first order svd approximation.    If the random variables are comonotonic then $Q=11'$ and $Q-I$ has ones everywhere except on the diagonal where it is zero.   The vector of row means is then $p^{-1}(p-1)1$
 
Systemic risk in the system is measured with $b'k=d_1(u_1'v_1)$.   In the case of comonotonic  random variables $Q=11'$, $a=1$ and $x$, where $p$ is the number of variables,  $d_1=1$. 
 
Furthermore we may define quantities such as
$
u^- \equiv \var_q(v|u\le q)
$
measuring the impact of a non distressed state in $v$.  For brevity we do not dwell on these constructs in this writeup although the ramifications and potential uses of these constructs will be  investigated in the research.



\section{Econometric implementation}

The above development sets out our proposed  broad  framework for linking bivariate copulas and marginals to external variables and shocks study the impact of the same on stresses within the system and the contagious effects of crises.   Proposed econometric analysis will implement and extend  \cite{brownlees2010volatility}.

\section{Data}

We will employ publicly available data as published by APRA and other regulators.

\section*{References}
\bibliography{/Applications/Tex/piet2}


\end{document}
