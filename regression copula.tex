\documentclass[authoryear]{elsarticle}
\usepackage{latexsym}
%\usepackage{rotate}
\usepackage{graphics}
%\usepackage{amsmath}
\usepackage{comment}
\bibliographystyle{chicago}



\newcommand{\logit}{\mathrm{logit}}
\newcommand{\I}{\mathrm{I}}
\newcommand{\E}{\mathrm{E}}
\newcommand{\p}{\mathrm{P}}
\newcommand{\e}{\mathrm{e}}
\newcommand{\vecm}{\mathrm{vec}}
\newcommand{\kp}{\otimes}
\newcommand{\diag}{\mathrm{diag}}
\newcommand{\cov}{\mathrm{cov}}
\newcommand{\eps}{\epsilon}
\newcommand{\ep}{\varepsilon}
\newcommand{\obdots}{\ddots}    % change this later
\newcommand{\Ex}{{\cal E}}
\newcommand{\rat}{{\frac{c_{ij}}{c_{i,j-1}}}}
\newcommand{\rmu}{m}
\newcommand{\rsig}{\nu}
\newcommand{\fd}{\mu}
\newcommand{\tr}{\mathrm{tr}}
\newcommand{\cor}{\mathrm{cor}}
\newcommand{\bx}[1]{\ensuremath{\overline{#1}|}}
\newcommand{\an}[1]{\ensuremath{a_{\bx{#1}}}}

\newcommand{\bi}{\begin{itemize}}
\newcommand{\ei}{\end{itemize}}

\renewcommand{\i}{\item}
\newcommand{\sr}{\ensuremath{\mathrm{SRISK}}}
\newcommand{\cs}{\ensuremath{\mathrm{CS}}}
\newcommand{\cri}{\ensuremath{\mathrm{Crisis}}}
\newcommand{\var}{\ensuremath{\mathrm{VaR}}}
\newcommand{\covar}{\ensuremath{\mathrm{CoVaR}}}
\newcommand{\med}{\ensuremath{\mathrm{m}}}
\newcommand{\de}{\mathrm{d}}
\renewcommand{\v}{\ensuremath{\mathrm{v}_q}}
\newcommand{\m}{\ensuremath{\mathrm{m}}}
\newcommand{\tvar}{\ensuremath{\mathrm{TVaR}}}



\newcommand{\eref}[1]{(\ref{#1})}
\newcommand{\fref}[1]{Figure \ref{#1}}
\newcommand{\sref}[1]{\S\ref{#1}}
\newcommand{\tref}[1]{Table \ref{#1}}
\newcommand{\aref}[1]{Appendix \ref{#1}}




\newcommand{\cq}{\ , \qquad}
\renewcommand{\P}{\mathrm{P}}
\newcommand{\Q}{\mathrm{Q}}

\newcommand{\Vx}{{\cal V}}
\newcommand{\be}[1]{\begin{equation}\label{#1}}
\newcommand{\ee}{\end{equation}}





\begin{document}

\section{Regression dependence model}

The regression dependence model is
$$
z_t=\alpha_t+\beta q_t + s\eps_t \;.
$$
Then
$$
z_t-z_{t-1}=(\alpha_t-\alpha_{t-1})+\beta (q_t-q_{t-1}) + s(\eps_t-\eps_{t-1}) \;,
$$
yielding the variance breakdown at $[t,t+1]$:
$$
V_t^z=V_t^r+V_t^\eps 
$$
where $V_t^z$, $V_t^r$ and $V_t^\eps$ are total, regression and error sum of squares at $[t,t+1]$, defined as
$$
V_t^z \equiv \E\left\{(z_t-z_{t-1})^2\right\} \cq V_t^\eps \equiv \E\left\{(\eps_t-\eps_{t-1})\right\}^2 \;,
$$
$$ 
V_t^r \equiv \E\left\{(\alpha_t-\alpha_{t-1})^2\right\}+\beta^2 (q_t-q_{t-1})^2
+2\beta (q_t-q_{t-1}) \E(\alpha_t-\alpha_{t-1}) \;.
$$
A measure of ``local dependence" at $[t,t+1]$ is the proportion of total variation explained by regression as opposed to error:
$$
\rho_t \equiv \frac{V_t^r}{V_t^z} \;,
$$
which is higher if $\alpha_t$ and $q_t$ ``move faster" and ``together" relative to $\eps_t$.

Questions:
\bi
\i Does specifying $\rho_t$ for all $t$ lead to a specification of $\alpha_t$?

\i Should $s$ vary with $t$? Otherwise how do we model the case of say weak lower tail dependence, since $s$ will need to be large for small $t$? 

\i Is it possible to fit $z_t=\alpha_t+\beta q_t + s_t\eps_t$ where $s_t$ is unknown and varies smoothly, similar to $\alpha_t$? Maybe $s_t$ varies inversely with $\alpha_t$.

\i Maybe should try fitting the model to data with asymmetric dependence and see if it fits properly.
\ei


\section{Factor copula model}

A factor copula model is
$$
z_t=\phi(q_t)+s\eps_t
$$
where $\phi$ is a transformation and $\phi(q_t)$ corresponds to $\alpha_t+\beta q_t$ in the regression above.

The first order difference is
$$
z_t-z_{t-1}=\phi(q_t)-\phi(q_{t-1})+(\eps_t-\eps_{t-1})
\sim \phi'(q_t)(q_t-q_{t-1}) +(\eps_t-\eps_{t-1}) \;.
$$
hence the variance decomposition at $[t,t+1]$ is
$$
\E\left\{(z_t-z_{t-1})^2\right\} = \{\phi(q_t)-\phi(q_{t-1})\}^2  + \E\left\{(\eps_t-\eps_{t-1})\right\}^2 \;.
$$
The issue here is that the marginal distribution of $z_t$ is going to be ``out of control." Maybe $s$ also needs to vary with $t$?

\section*{References}
\bibliography{piet2}


\end{document}
