\documentclass[authoryear]{elsarticle}
\usepackage{latexsym}
%\usepackage{rotate}
\usepackage{graphics}
\usepackage{amsmath}
\usepackage{amssymb}
\usepackage{comment}
\bibliographystyle{chicago}


\newcommand{\logit}{\mathrm{logit}}
\newcommand{\I}{\mathrm{I}}
\newcommand{\E}{\mathrm{E}}
\newcommand{\p}{\mathrm{P}}
\newcommand{\e}{\mathrm{e}}
\newcommand{\vecm}{\mathrm{vec}}
\newcommand{\kp}{\otimes}
\newcommand{\diag}{\mathrm{diag}}
\newcommand{\cov}{\mathrm{cov}}
\newcommand{\eps}{\epsilon}
\newcommand{\ep}{\varepsilon}
\newcommand{\obdots}{\ddots}    % change this later
\newcommand{\Ex}{{\cal E}}
\newcommand{\rat}{{\frac{c_{ij}}{c_{i,j-1}}}}
\newcommand{\rmu}{m}
\newcommand{\rsig}{\nu}
\newcommand{\fd}{\mu}
\newcommand{\tr}{\mathrm{tr}}
\newcommand{\cor}{\mathrm{cor}}
\newcommand{\bx}[1]{\ensuremath{\overline{#1}|}}
\newcommand{\an}[1]{\ensuremath{a_{\bx{#1}}}}

\newcommand{\bi}{\begin{itemize}}
\newcommand{\ei}{\end{itemize}}

\renewcommand{\i}{\item}
\newcommand{\sr}{\ensuremath{\mathrm{SRISK}}}
\newcommand{\cs}{\ensuremath{\mathrm{CS}}}
\newcommand{\cri}{\ensuremath{\mathrm{Crisis}}}
\newcommand{\var}{\ensuremath{\mathrm{VaR}}}
\newcommand{\covar}{\ensuremath{\mathrm{CoVaR}}}
\newcommand{\med}{\ensuremath{\mathrm{m}}}
\newcommand{\de}{\mathrm{d}}
\renewcommand{\v}{\ensuremath{\mathrm{v}_q}}
\newcommand{\m}{\ensuremath{\mathrm{m}}}
\newcommand{\tvar}{\ensuremath{\mathrm{TVaR}}}
\newcommand{\mes}{\ensuremath{\mathrm{MES}}}
\newcommand{\es}{\ensuremath{\mathrm{ES}}}
\newcommand{\lmes}{\ensuremath{\mathrm{LRMES}}}
\newcommand{\coes}{\ensuremath{\mathrm{CoES}}}
\newcommand{\cat}{\ensuremath{\mathrm{CATFIN}}}
\newcommand{\sged}{\ensuremath{\mathrm{SGED}}}
\newcommand{\gpd}{\ensuremath{\mathrm{GPD}}}

\newcommand{\eref}[1]{(\ref{#1})}
\newcommand{\fref}[1]{Figure \ref{#1}}
\newcommand{\sref}[1]{\S\ref{#1}}
\newcommand{\tref}[1]{Table \ref{#1}}
\newcommand{\aref}[1]{Appendix \ref{#1}}




\newcommand{\cq}{\ , \qquad}
\renewcommand{\P}{\mathrm{P}}
\newcommand{\Q}{\mathrm{Q}}

\newcommand{\Vx}{{\cal V}}
\newcommand{\be}[1]{\begin{equation}\label{#1}}
\newcommand{\ee}{\end{equation}}




\begin{document}

\section*{Measuring  systemic risk}

\section{$\sr$}

$\sr$ is a Systemic Risk measure based on the Marginal Expected Shortfall [$\mes$] due to  \cite{acharya2012aer}. The expected capital shortfall of financial an institution is
$$S=\left|k\frac{D+W}{W}-(1-k)M-1\right|^+W\ ,
$$
where $k$ is a prudential capital requirement, $W$ is market value of equity; $D$ is book value of debt and
$$
M=\E_t(R_{j,t+1:t+h}|R_{m,t+1:t+h}<C)\ ,
$$ 
is the long run $\mes$, 
where $R_{jt}$ is equity return on financial institution $j$, $R_{mt}$ is return on the financial system, $C$ is crisis threshold return.

Aggregate systemic risk of financial system is $$\sum_{j=1}^{N}S_{jt}.$$

\subsection{\cite{brownlees2015}}

\bi
\i Data: 95 large US financial firms; daily equity returns and market capitalization from CRSP; quarterly book values for equity and debt from Compustat
\i Period: 2000-2012 
\i Econometric methods:  
\bi
\i estimate Dynamic Conditional Correlation (DCC) model for firm and market with threshold GARCH volatilities
\begin{equation*}
\begin{bmatrix}
R_{jt}\\ 
R_{mt} 
\end{bmatrix} \sim \mathcal{D}\left ( 0, 
\begin{bmatrix}
\sigma_{jt}^2 & \rho\sigma_{jt}\sigma_{mt}\\ 
\rho\sigma_{jt}\sigma_{mt} & \sigma_{mt}^2 
\end{bmatrix}\right )
\end{equation*}
\i obtain $\lmes$ using Monte Carlo simulation of returns based on model estimates. Use $S$ repetitions and compute 
$$\lmes_{jt}=\frac{\sum_{s=1}^{S}R_{jt+1:t+h}^{s} \times I(R_{mt+1:t+h}^{s} < C)}{\sum_{s=1}^{S}I(R_{mt+1:t+h}^{s}<C)}$$
where $I(x)=1$ if true and 0 otherwise.
\ei

\i Results:
\bi
\i $\sr$ successfully identifies systemically risky firms during the GFC
\i $\sr$ has predictive value: predicts capital injections by Fed Reserve
\i aggregate $\sr$ provides early warning of declines in industrial production and higher unemployment  
\ei 
\ei


\subsection{\cite{Engle2014}}
\bi
\i Data: 196 large financial institutions in Europe: banks, insurance companies, financial services firms and real estate firms; daily equity returns and market capitalization from Datastream; quarterly book values for equity and debt from Compustat; world and Europe equity indexes from MSCI
\i Period: 1990-2012 
\i Econometric methods:  
\bi
\i estimate multi-factor, time varying model of returns using Dynamical Conditional Betas,
\cite{Engle2014dcb}
\i model volatility of errors using univariate asymmetric GARCH model
\i use skewed $t$ distribution for marginals of the innovations and a \textit{t} copula for the dependence structure between the innovations
\i estimation proceeds recursively from an international model [World and European indexes]; then country models [add in respective country index and use the parameters from the international model] then firm model [include firm returns and use parameters from country model]  
\i directly estimate $\lmes$
\bi
\i simulate forward 125 days returns from model estimates
\i use $S=50,000$ draws to compute
$$\lmes_{jt}=\frac{\sum_{s=1}^{S}R_{jt+1:t+h}^{s} \times I(R_{mt+1:t+h}^{s} < -40\%)}{\sum_{s=1}^{S}I(R_{mt+1:t+h}^{s}<-40\%)}$$
where $I(x)=1$ if true and 0 otherwise.
where $I(x)=1$ if true and 0 otherwise.
\ei 
\ei 

\i Results:
\bi
\i rank firms, firm types and countries by $\sr$: banks contribute 83\% insurance companies 15\% of systemic risk in Europe; highest countries - France and UK contribute 52\%
\i aggregate $\sr$ Granger-causes industrial production and business confidence index
\i $\sr$ is positively related to changes in 3-month interbank rate but not significantly related to stock market return and volatility.
\ei
\ei





\section{$\covar$}

\cite{adrian2011} propose $\covar$ which is the $\var$ of the financial system $m$ when financial institution $j$ is in distress which is operationally defined as its $\var(q)$ level. 
$$\text{Pr}(R_t^m\leq\covar_{qt}^{m|j}|R_t^j=\var_{qt}^j)=q.$$
Increased risk to system $m$ when financial institution $j$ is in distress 
$$\Delta \covar_{qt}=\covar_{qt}^{m|j}-\covar_{qt}^{m|b^j}$$ 
where $b^j$ denotes the benchmark state for $j$ and equals its median return.

\cite{Girardi2013} suggest to condition on when financial institution $j$ is at its $\var$ level at best:
$$\text{Pr}(R_t^m\leq \covar_{qt}^{m|j}|R_t^j\leq \var_{qt}^j)=q.$$ They define the benchmark state $b^j$ as a one standard deviation around the mean
$$\mu_t^j - \sigma_t^j \leq R_t^j \leq \mu_t^j + \sigma_t^j.$$
and measure the percentage systemic risk contribution of $j$ by
$$\Delta \covar_{q,t}^{m|j}=100\times (\covar_{qt}^{m|j}-\covar_{qt}^{m|b^j})/\covar_{qt}^{m|b^j}.$$  \cite{Girardi2013} argue their measure improves on \cite{adrian2011} as it
\bi
\i is more general and relevant as it takes into account severity of tail losses
\i facilitates back-testing of $\covar$ using standard back-tests for $\var$.
\ei
\cite{Mainik2012} also show the \cite{Girardi2013} measure is better 
\bi
\i it is consistent with respect to the dependence parameter unlike the \cite{adrian2011} measure which is not, e.g.in a bivariate Gaussian model, the \cite{adrian2011} measure is decreasing in the correlation! 
\i consistency property also extends to Conditional Expected Shortfall [$\coes$]. 
\ei 


\subsection{\cite{adrian2011}}
\bi
\i Data: weekly equity returns of 357 US bank holding companies from CRSP; quarterly balance sheet data from Compustat; state variables: VIX, liquidity spread; change in T-bill rate; change in slope of yield curve; change in credit spread; controls - market equity return; return on real estate sector.
\i Period:  1986-2010
\i Econometric methods:  
\bi
\i use quantile regressions of asset returns to estimate $CoVaR$ 
\i add lagged state variables for conditional results
\i construct out-of-sample forward $\Delta \covar$ by projecting $\Delta \covar$ on lagged size, leverage, maturity mismatch and industry dummies
\ei
\i Results:
\bi
\i $\var$ and $\Delta \covar$ are weakly related in the cross section but strongly related over time
\i $\Delta \covar$ larger for firms with higher leverage, more maturity mismatch and larger size
\i strong negative correlation between contemporaneous $\Delta \covar$ and forward $\Delta \covar$
\ei
\ei 


\subsection{\cite{Girardi2013}}
\bi
\i Data:  estimate $CoVaR$ and conduct back-tests using daily equity returns of 74 large US financial firms 
\i Period:  2000-20087 
\i Econometric methods:  
\bi
\i bivariate GARCH-DCC models assuming (i) bivariate Gaussian (ii) bivariate skewed-$t$ distributions
\i 3-step procedure: 1. estimate individual $\var$ using univariate GARCH; 2. estimate bivariate density for returns using bivariate GARCH; 3. numerically solve $\covar$s
$$\int_{-\infty}^{\covar_{qt}^{m|j}}\int_{-\infty}^{\var_{qt}^{j}}pdf_t(x,y)dydx=q^2$$
$$\int_{-\infty}^{\covar_{qt}^{m|b^j}}\int_{\mu_t^j - \sigma_t^j}^{\mu_t^j + \sigma_t^j} pdf_t(x,y)dydx=p_t^j q.$$
\ei

\i Results:
\bi
\i backtests favour skewed-$t$ distribution 
\i rank of contribution to systemic risk: depositary institutions [largest]; broker-dealers; insurance companies; non-depositary institutions [smallest]
\i $\var$ and $\Delta \covar$ are weakly related cross sectionally and over time
\i all industry groups showed substantial increase in pre crisis $\Delta \covar$ 
\i $\Delta \covar$ is positively related to size and equity beta, and to leverage during down markets.
\ei
\ei 



\section{$\cat$}

\cite{Allen2012} create a macroindex of systemic risk designated $\cat$  which is the arithmetic average of three $\var$ measures using the Generalized Pareto distribution [$\gpd$], skewed generalized error distribution [$\sged$] and a nonparametric approach being the relevant quantile of the empirical distribution. 

$$\var_{\gpd} =\mu + \left(\frac{\sigma}{\xi} \right) \left[  \left(\frac{\alpha N}{n} \right)^{-\xi} -1\right]  $$ where $\mu,\sigma,\xi$ are location, scale and shape parameters. $\alpha$ is loss probability level, $n$ is number of extremes and $N$ is total data points.

Obtain $\var_{\sged}$ by solving numerically
$$\int_{-\infty}^{\var_{\sged(\alpha)}}f_{\mu,\sigma,\kappa,\lambda}(z)dz=\alpha$$
where $f$ is $\sged$ probability density function; $\kappa$ controls height and tails of the density and $\lambda$ is skewness.

\cite{Allen2012} also create analogous measures for expected shortfall but report that the predictability results are similar.
 
\subsection{\cite{Allen2012}}
\bi
\i Data:  monthly returns and market capitalization for US financial companies; Chicago Fed National Activity Index [CFNAI] and alternative macro-economic performance indicators; collect similar stock price data for EU and Asia and GDP growth rates 
\i Period:  1973-2009  
\i Econometric methods:  
\bi
\i use maximum likelihood to estimate $\gpd$ and $\sged$ parameters from monthly returns
\i estimate predictive auto-regressions of CFNAI using $\cat$ as a predictor plus other control variables
\ei

\i Results:
\bi
\i $\cat$ is negatively related to the future CFNAI for 1- to 6-month ahead forecast horizons 
\i this predictability comes from banks
\i results are robust to alternative measures of economic performance
\i regional $\cat$ also has predictive power for GDP growth in EU and Asia
\ei
\ei

\section*{References}

\bibliography{cifrEmpirics}


\end{document}

\subsection{\cite{}}
\bi
\i Data:   
\i Period:  2000-2007 
\i Econometric methods:  
\bi
\i 
\i 
\ei

\i Results:
\bi
\i 
\i 
\ei
\ei


\subsection{$SRISK$}


\ei




\subsection{$CoVaR$}

Estimate $\covarab$. 

Data: d

Period: 1986-2010 

Estimation:
\bi

\ei 

Results:
\bi
\i
\ei


\cite{Girardi2013}

Estimation:
\bi

\ei 

Results:
\bi

\ei 




\cite{Huang2009}

\cite{Allen2012}

\cite{adrian2011} 
Purpose: propose a countercyclical, forward looking systemic risk measure 

Data: daily equity returns of large US financial firms; quarterly balance sheet data; state variables: VIX, liquidity spread; change in T-bill rate; change in slope of yield curve; change in credit spread; controls - market equity return; return on real estate sector

Period: 1986-2010 

Definitions:
$CoVaR$ 

Increased risk to system $m$ when firm $i$ is at its $VaR$ level (in distress) is 
$$\Delta CoVaR_t=CoVaR_{i,t}^{q,q}-CoVaR_{i,t}^{q,0.5}$$

Estimation:
\bi
\i use quantile regressions of asset returns to estimate $CoVaR$
\i add lagged state variables for conditional results
\ei 



 is contribution of institution $i$ to system ES
$$MES_{it}(C)=E_{t-1}(r_{it}|r_{mt}<C)$$

Systemic Expected Shortfall is amount capital [$W$] drops below its target level of $k$ times assets [$A$], conditional on a systemic crisis which occurs when aggregate capital is less than $k$ times aggregate assets
$$\frac{SES_{it}}{W_{it}}=kL_{it}-1-E_{t-1}(r_{it}|\sum_{i=1}^N W_{it}<k \sum_{i=1}^N A_{it})$$
SES is linearly related to MES:
$$SES_{it}=(kL_{it}-1+\theta MES_{it}+\Delta_i)W_{it}$$ where $\theta,\Delta$ are constants.

SRISK extends MES by accounting for liabilities and size of financial institution
$$SRISK_{it}=\max[0,(kL_{it}-1+(1-k)LRMES_{it})W_{it}]$$

SES and SRISK almost identical; choose SRISK.

Relations:
Assume a bivariate GARCH-DCC model

$$MES_{it}(\alpha)=\beta_{it}ES_{mt}(\alpha)$$ where $\beta_{it}=\rho_{it}\sigma_{it}/\sigma_{mt}.$

$$SRISK_{it}= \max[0,(kL_{it}-1+(1-k)LRMES_{it})W_{it}]$$
\cite{acharya2012aer} propose $LRMES_{it}\simeq 1-\exp(18\times MES_{it})$ [why 18? convert daily threshold of 2\% to 6 month fall of 40\%] so 
$$SRISK_{it}\simeq \max[0,(kL_{it}-1+(1-k)\exp(18\times \beta_{it} \times ES_{mt}(\alpha))W_{it}]$$

$$\Delta CoVaR_{it}(\alpha)=\gamma_{it}[VaR_{it}(\alpha)-VaR_{it}(0.5)]$$ where $\gamma_{it}=\rho_{it}\sigma_{mt}/\sigma_{it}.$

\subsection{SRISK}

\subsection{CoVaR}
\subsection{\cite{adrian2011}}

Purpose: propose a countercyclical, forward looking systemic risk measure 

Data: daily equity returns of large US financial firms; quarterly balance sheet data; state variables: VIX, liquidity spread; change in T-bill rate; change in slope of yield curve; change in credit spread; controls - market equity return; return on real estate sector

Period: 1986-2010 

Definitions:
$CoVaR$ 
$$\text{Pr}(R_{m,t}<-CoVaR_{i,t}^{q,p}|R_{i,t}=-VaR_{i,t}^p)=q$$
Increased risk to system $m$ when firm $i$ is at its $VaR$ level (in distress) is 
$$\Delta CoVaR_t=CoVaR_{i,t}^{q,q}-CoVaR_{i,t}^{q,0.5}$$

Estimation:
\bi
\i use quantile regressions of asset returns to estimate $CoVaR$
\i add lagged state variables for conditional results
\ei 




\subsubsection{CoVaR[AB]}

AB 

\subsubsection{CoVaR}

Explain why this is better 
 
\subsection{CATFIN}


\subsection{}




\subsection{



\subsection{\citet{acharya2012aer}}

\cite{acharya2012aer}

\subsection{\citep{acharya2012wp}}

\cite{acharya2012wp}




\subsection{\cite{Engle31032014}}

Purpose: measure systemic risk for large financial institutions in Europe: banks, insurance companies, financial services firms and real estate firms.

Data: individual equity returns; index returns for world, Europe and country; financial leverage

Period: 1990-2012

Definitions:

Long run marginal expected shortfall [expected loss conditional on extreme event]:
$$LRMES_{i,t:t+T}=-E_t[R_{i,t:t+T} | R_{M,t:t+T} \leq -40\%]$$
where $R$ is equity return; $40\%$ is arbitrary definition of crisis and equal to worst 6-month market fall over last decade.

Capital shortfall
$$CS_{i,t:t+T}=\{\theta(L_{i,t}-1)-(1-\theta )E_t(1-LRMES_{i,t:t+T})\}W_{i,t}$$
where $L$ is financial leverage; $\theta$ is prudential ratio of equity to assets; $W$ is market value of equity.

Systemic risk [positive capital shortfall]
$$SRISK_{i,t:t+T}=\max(0,CS_{i,t:t+T})$$

Industry level [value weighted sum]:
$$LRMES_{F,t:t+T}=\sum_{i=1}^{N}w_{i,t}LRMES_{i,t:t+T}$$

Econometric methods: 
\bi
\i estimate multi-factor, time varying model of returns using Dynamical Conditional Betas,
\cite{Engle2014}
\i model volatility of errors using univariate asymmetric GARCH model
\i use skewed $t$ distribution for marginals of the innovations and a \textit{t} copula for the dependence structure between the innovations
\i estimation proceeds recursively from an international model [World and European indexes]; then country models [add in respective country index and use the parameters from the international model] then firm model [include firm returns and use parameters from country model]  
\i directly estimate $LRMES$
\bi
\i simulate forward 125 days returns from model estimates
\i use $S=50,000$ draws to compute
$$LRMES_{i,t:t+T}^{(W)}=-\frac{\sum_{s=1}^{S}R_{i,t:t+T}^{(s)} \times I(R_{W,t:t+T}^{(s)}\leq -40\%)}{\sum_{s=1}^{S}I(R_{W,t:t+T}^{(s)}\leq -40\%)}$$
where $I(x)=1$ if true and 0 otherwise.
\ei 
\ei 

Suggestions:  

\subsection{\cite{Corvasce2013}}

\subsection{\cite{Benoit2013}}

Purpose: show how several popular systemic risk measures are transformations of market risk measures and derive conditions under which these measures provide similar rankings of systemically important financial institutions

Data: daily returns for large US financial firms; quarterly book values for debt

Period: 2000-2010

Definitions:

Marginal Expected Shortfall is contribution of institution $i$ to system ES
$$MES_{it}(C)=E_{t-1}(r_{it}|r_{mt}<C)$$

Systemic Expected Shortfall is amount capital [$W$] drops below its target level of $k$ times assets [$A$], conditional on a systemic crisis which occurs when aggregate capital is less than $k$ times aggregate assets
$$\frac{SES_{it}}{W_{it}}=kL_{it}-1-E_{t-1}(r_{it}|\sum_{i=1}^N W_{it}<k \sum_{i=1}^N A_{it})$$
SES is linearly related to MES:
$$SES_{it}=(kL_{it}-1+\theta MES_{it}+\Delta_i)W_{it}$$ where $\theta,\Delta$ are constants.

SRISK extends MES by accounting for liabilities and size of financial institution
$$SRISK_{it}=\max[0,(kL_{it}-1+(1-k)LRMES_{it})W_{it}]$$

SES and SRISK almost identical; choose SRISK.

Relations:
Assume a bivariate GARCH-DCC model

$$MES_{it}(\alpha)=\beta_{it}ES_{mt}(\alpha)$$ where $\beta_{it}=\rho_{it}\sigma_{it}/\sigma_{mt}.$

$$SRISK_{it}= \max[0,(kL_{it}-1+(1-k)LRMES_{it})W_{it}]$$
\cite{acharya2012aer} propose $LRMES_{it}\simeq 1-\exp(18\times MES_{it})$ [why 18? convert daily threshold of 2\% to 6 month fall of 40\%] so 
$$SRISK_{it}\simeq \max[0,(kL_{it}-1+(1-k)\exp(18\times \beta_{it} \times ES_{mt}(\alpha))W_{it}]$$

$$\Delta CoVaR_{it}(\alpha)=\gamma_{it}[VaR_{it}(\alpha)-VaR_{it}(0.5)]$$ where $\gamma_{it}=\rho_{it}\sigma_{mt}/\sigma_{it}.$


\subsection{\cite{Girardi2013}}


\subsection{more}

indicators of distress: equity return losses; CDS rates

\section*{References}

\bibliography{cifrEmpirics}


\end{document}

\cite{Huang2009}

\cite{Allen2012}

\subsection{Relations}

\cite{Benoit2013} show how several popular systemic risk measures are transformations of market risk measures and derive conditions under which these measures provide similar rankings of systemically important financial institutions.

Assume a bivariate GARCH-DCC model:
$$\mes_{jt}=\beta_{jt}\es_{st}$$ where $\beta_{jt}=\rho_{jt}\sigma_{jt}/\sigma_{st}.$
\cite{acharya2012aer} propose $LRMES_{jt}\simeq 1-\exp(18\times \mes_{jt})$ [why 18? convert daily threshold of 2\% to 6 month fall of 40\%] so

$$\sr_{jt}\simeq \min[0,(E_{jt}\exp(18\times \beta_{jt} \times \es_{st})-k(E_{jT}+D_{jT})]$$

$$\Delta \covar_{it}(\alpha)=\gamma_{it}[\var_{it}(\alpha)-\var_{it}(0.5)]$$ where $\gamma_{it}=\rho_{it}\sigma_{mt}/\sigma_{it}.$