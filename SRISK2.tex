\documentclass[authoryear]{elsarticle}
\usepackage{latexsym}
%\usepackage{rotate}
\usepackage{graphics}
\usepackage{amsmath}
\usepackage{amssymb}
\usepackage{comment}
\bibliographystyle{chicago}



\newcommand{\logit}{\mathrm{logit}}
\newcommand{\I}{\mathrm{I}}
\newcommand{\E}{\mathrm{E}}
\newcommand{\p}{\mathrm{P}}
\newcommand{\e}{\mathrm{e}}
\newcommand{\vecm}{\mathrm{vec}}
\newcommand{\kp}{\otimes}
\newcommand{\diag}{\mathrm{diag}}
\newcommand{\cov}{\mathrm{cov}}
\newcommand{\eps}{\epsilon}
\newcommand{\ep}{\varepsilon}
\newcommand{\obdots}{\ddots}    % change this later
\newcommand{\Ex}{{\cal E}}
\newcommand{\rat}{{\frac{c_{ij}}{c_{i,j-1}}}}
\newcommand{\rmu}{m}
\newcommand{\rsig}{\nu}
\newcommand{\fd}{\mu}
\newcommand{\tr}{\mathrm{tr}}
\newcommand{\cor}{\mathrm{cor}}
\newcommand{\bx}[1]{\ensuremath{\overline{#1}|}}
\newcommand{\an}[1]{\ensuremath{a_{\bx{#1}}}}

\newcommand{\bi}{\begin{itemize}}
\newcommand{\ei}{\end{itemize}}

\renewcommand{\i}{\item}
\newcommand{\sr}{\ensuremath{\mathrm{SRISK}}}
\newcommand{\cs}{\ensuremath{\mathrm{CS}}}
\newcommand{\cri}{\ensuremath{\mathrm{Crisis}}}
\newcommand{\var}{\ensuremath{\mathrm{VaR}}}
\newcommand{\covar}{\ensuremath{\mathrm{CoVaR}}}
\newcommand{\med}{\ensuremath{\mathrm{m}}}
\newcommand{\de}{\mathrm{d}}
\renewcommand{\v}{\ensuremath{\mathrm{v}_q}}
\newcommand{\m}{\ensuremath{\mathrm{m}}}
\newcommand{\tvar}{\ensuremath{\mathrm{TVaR}}}



\newcommand{\eref}[1]{(\ref{#1})}
\newcommand{\fref}[1]{Figure \ref{#1}}
\newcommand{\sref}[1]{\S\ref{#1}}
\newcommand{\tref}[1]{Table \ref{#1}}
\newcommand{\aref}[1]{Appendix \ref{#1}}




\newcommand{\cq}{\ , \qquad}
\renewcommand{\P}{\mathrm{P}}
\newcommand{\Q}{\mathrm{Q}}

\newcommand{\Vx}{{\cal V}}
\newcommand{\be}[1]{\begin{equation}\label{#1}}
\newcommand{\ee}{\end{equation}}




\begin{document}

% Title of paper
\title{Systemic risk and contagion effects in Australian financial institutions and sectors}
% List of authors, with corresponding author marked by asterisk
\author{Piet de Jong,  Geoff Loudon and Weihao Choo \\[4pt]
% Author addresses
\textit{Department of Applied Finance and Actuarial Studies\\ Macquarie University, Sydney, NSW 2109.}
\\[2pt]
%E-mail address for correspondence
{piet.dejong@mq.edu.au}}

% Running headers of paper:
\markboth%
% First field is the short list of authors
{De Jong}
% Second field is the short title of the paper
{Systemic risk}

\maketitle



\section{Simplified notation}

Suppose the system is currently at time $t$ and the capital shortfall for firm $i$ at time $t+h$ is
$$
c_i \equiv  kd_i -(1-k)w_i(1+r_i)
$$
where $k$ is the prudential requirement, $d_i$ and $w_i$ are debt and equity at time $t$, and $r_i$ is the return on equity from time $t$ to $t+h$. The only random quantity in the above definition is $r_i$. The total capital shortfall in the system is
$$
c_m = \sum_i c_i = k\sum_i d_i -(1-k)\sum_i w_i(1+r_i) = k d_m - (1-k)w_m(1+r_m) 
$$
where
$$
d_m \equiv \sum_i d_i \cq w_i \equiv \sum_i w_i \cq r_m \equiv \frac{\sum_i w_ir_i}{\sum_i w_i}
$$
are the overall debt and equity for the market at time $t$ and the market return on equity from time $t$ to $t+h$. Note it is assumed that capital surpluses in certain firms are used to offset shortfalls in other firms, that is a diversification benefit is allowed in the system.

The systemic risk contribution by firm $i$ at time $t+h$ is the expected shortfall in systemic event $r_m<a$
$$
\mathrm{SRISK}_i = \E(c_i|r_m<a) = kd_i -(1-k)w_i\{1+\E(r_i|r_m<a)\} \;,
$$
and is computed from the joint probability distribution of $(r_i,r_m)$. Replace the systemic event with one having a probabilistic interpretation: $u_m < a$ where $0<a<1$ and $u_m$ is the percentile rank of $r_m$. The refined systemic event $u < a$ captures the worst $a$ of possible market returns and has probability $a$. In addition the threshold $a$ is simply the percentile rank of the original threshold. Hence the revised systemic risk of firm $i$ is
$$
\mathrm{SRISK}_i \equiv \E(c_i|u_m<a) = kd_i -(1-k)w_i\{1+\E(r_i|u_m<a)\} \;.
$$
In addition the aggregate systemic risk is
$$
\mathrm{SRISK}_m \equiv \sum_i \mathrm{SRISK}_i = \sum_i \E(c_i|u_m<a) = \E(c_m|u_m<a)
$$
$$
=kd_m -(1-k)w_m\{1+\E(r_m|u_m<a)\} \;.
$$
Note the original definition of aggregate systemic risk is the sum of positive systemic risks of individual firms
$$
\sum_i \left(\mathrm{SRISK}_i\right)_+
$$
and does not result in the above simplification. The original definition assumes no diversification: capital surpluses cannot cover shortfalls. The systemic risk of firm $i$ is positive, indicating capital shortfall, if and only if
$$
\E(r_i|u_m<a) < \frac{kd_i}{(1-k)w_i} - 1
$$




\section{Alternative measures of SRISK}

Measuring systemic risk at a single systemic event $u_m<a$ presents difficulties as to an appropriate choice of $a$. This difficulty is overcome by considering a range of systemic events with different aversion weights attached. Taking a weighted average of conditional tail expectations is akin to the construction of spectral risk measures, and results are weighted premiums or risks already discussed extensively in the literature.

Note a weighted average of the conditional tail expected return of firm $i$ is
$$
\int_0^1 \E(r_i|u_m<a) w(a) \de a = \int_0^1 \frac{w(a)}{a} \left\{ \int_0^a \E(r_i|u_m=v) \de v \right\}  \de a
$$
$$
=\int_0^1 \E(r_i|u_m=v) \int_v^1  \frac{w(a)}{a}     \de a \de v = \E\left\{\E(r_i|u_m)\phi(u_m)\right\}=\E\{r_i\phi(u_m)\}
$$
where
$$
\phi(v)\equiv \int_v^1 \frac{w(c)}{c}     \de c  
\cq \int_0^1 \phi(v) \de v = \int_0^1 \int_v^1 \frac{w(c)}{c} \de c \de v = \int_0^1 \frac{w(c)}{c} \int_0^c \de v \de c = 1\;.
$$
Hence the result is an aversion expected return of firm $i$ with aversion weights $\phi(u_m)$ or $\E\{\phi(u_m)|r\}$ integrating to one provided the same applies to $w(a)$. Note $\phi$ is a decreasing function which is straightforward from its definition. In addition the aversion adjusted expectation of $r_i$ can be written as
$$
\E\{r_i\phi(u_m)\} = \E(r_i)+\cov\{r_i,\phi(u_m)\} \;,
$$
the original expectation plus a negative risk adjustment.

Applying the same weighted averaging to SRISK yields
$$
\widetilde{\mathrm{SRISK}}_i \equiv \int_0^1 \mathrm{SRISK}_i(a) w(a) \de a = \E\{c_i\phi(u_m)\} 
$$
$$
= kd_i-(1-k)w_i[1+\E\{r_i\phi(u_m)\}] \;,
$$
which is the aversion adjusted expected capital shortfall.



\section{Standardised SRISK and layer dependence}

Standardising SRISK by subtracting the unconditional expected capital shortfall and dividing by the same assuming maximum systemic risk yields a quantity which is independent of initial debt and equity. In addition standardised SRISK reflects the local dependence between an individual firm and the market at the selected threshold of the systemic event. Hence varying the threshold yields the dependence structure of individual firm and market returns from benign events all the way to tail events.

The unconditional expected capital shortfall for firm $i$ is $\E(c_i)=kd_i-(1-k)w_i\{1+\E(r_i)\}$, suggesting the standardised SRISK
$$
\mathrm{SRISK}_i^* \equiv \frac{\mathrm{SRISK}_i-\E(c_i)}{\max(\mathrm{SRISK}_i)-\E(c_i)}
=\frac{\E(r_i)-\E(r_i|u_m<a)}{\E(r_i)-\E(r_i|u_i<a)}
$$
$$
=\frac{\cov\{r_i,\overline{I}_a(u_m)\}}{\cov\{r_i,\overline{I}_a(u_i)\}}
=\frac{\cov\{r_i,I_a(u_m)\}}{\cov\{r_i,I_a(u_i)\}} 
$$
where $\overline{I}_a(z)\equiv (z<a)$ and $I_a(z)\equiv (z>a)$ are indicators. Maximum SRISK occurs if firm return is comonotonic with the market return or $u_i=u_m$, $u_i$ being the percentile rank of $r_i$. Standardised SRISK is hence layer dependence on the original scale, and mainly lies between 0 and 1 indicating independence and comonotonicity, respectively. Varying $a$ reveals the local dependence between firm and market returns at different thresholds, with $a\rightarrow 0$ yielding the lower extreme tail dependence in a market crash of maximum severity.

Performing the same standardising on $\widetilde{\mathrm{SRISK}}_i$ which is a weighted average of SRISK for firm $i$ across various thresholds yields
$$
\widetilde{\mathrm{SRISK}}_i^* = \frac{\cov\{r_i,\phi(u_m)\}}{\cov\{r_i,\phi(u_i)\}}   \;.
$$

\section{Standardised SRISK, correlation and beta}

Taking a weighted average of the layer dependence curve yields correlation. This implies a weighted average of standardised SRISKs for a firm across different thresholds is the overall correlation between firm and market returns. This correlation is related to the beta or systematic risk of the firm in the capital asset pricing model. Hence systemic risk relates specifically to a market crash or systemic event, whilst systematic risk relates to the average over all market return possibilities.

Since 
$$
\int_0^1 I_a(u_m) \de F^-_m(a)  = \int_0^{u_m} \de F^-_m(a) = r_m-\inf(r_m) \;,
$$
taking a weighted average of standardised SRISKs for firm $i$ yields
$$
\frac{\int_0^1 \mathrm{SRISK}_i(a) \cov\{r_i,I_a(u_i)\} \de F^-_m(a) }
{\int_0^1\cov\{r_i,I_a(u_i)\} \de F^-_m(a) }
=\frac{\int_0^1 \cov\{r_i,I_a(u_m)\} \de F^-_m(a) }
{\int_0^1\cov\{r_i,I_a(u_i)\} \de F^-_m(a) }
$$
$$
=\frac{\cov\{r_i,r_m-\inf(r_m)\}}{\cov\{r_i, r_m^*-\inf(r_m)\}}
=\frac{\cov(r_i, r_m)}{\cov(r_i, r_m^*)}
=\frac{\cor(r_i, r_m)}{\cor(r_i, r_m^*)} = \frac{\beta_i\sigma_m }{\sigma_i\cor(r_i, r_m^*)}
$$
where $r_m^*$ has the same marginal distribution as $r_m$ but is comonotonic with $r_i$ and
$$
\beta_i \equiv \frac{\cov(r_i,r_m)}{\cov(r_m,r_m)} = \cor(r_i,r_m)\frac{\sigma_i}{\sigma_m} 
$$
is the beta of firm $i$ under the capital asset pricing model, where $\sigma_i$ and $\sigma_m$ are volatilities of $r_i$ and $r_m$ respectively. Hence the above weighted average of SRISKs for firm $i$ is the scaled correlation between firm and market returns, and is directly proportional to the beta of the firm.





\section{Relative systemic risk contributions}

The contribution of firm $i$ to total systemic risk is originally defined as the positive expected capital shortfall relative to the same computed over all firms in the market:
$$
\frac{ (\mathrm{SRISK}_i)_+}{\sum_i(\mathrm{SRISK}_i)_+}  \;.
$$
A large proportion indicates a systemically important firm. 

The following improves the computation in two ways. Firstly firms with capital surplus are included, since they, especially those with large surpluses, are systemically important but in a positive way by dampening the impact from distressed firms. Secondly it is more appropriate to consider the departure from the unconditional expectation to differentiate firms which only experience capital shortfall during a market downturn as opposed to those who already have large shortfalls in average conditions. The revised contribution of firm $i$ to total systemic risk is hence
$$
\frac{\mathrm{SRISK}_i-\E(c_i)}{\mathrm{SRISK}_m-\E(c_m)}
=\frac{w_i }{w_m } \frac{\E(r_i)-\E(r_i|u_m<a)}{\E(r_m)-\E(r_m|u_m<a)} \;,
$$
the product of the equity of firm $i$ relative to total market equity and return movement of firm $i$ in a market downturn relative to overall market movement.




\section{Systemic risk}

If $d_{it}$ and $w_{it}$ are the debt and equity of a firm $i$ at time $t$ then the capital shortfall at time $t$ is 
\be{shortfall}
kd_{it}  - (1-k) w_{it} = kd_{it}\left(1-\e^{-\ell^*_{it}}\right)\ ,
\ee
where 
$$
\ell^*_{it} \equiv \ell_{it} + \ln \frac{k}{1-k}\cq \ell_{it} \equiv \ln\frac{d_{it}}{w_{it}}\ .
$$
The quantity $\ell^*_{it}$ is called Basel adjusted the log--leverage and $\ell^*_{it}>0$ implies capital shortfall \eref{shortfall} is positive. 


Basel II assumes $k=0.08$ implying $\ell_{it}^*=\ell_{it}-2.44$  and there is  positive  ``Basel shortfall"  if
$\ell_{it} > 2.44$.
As the log leverage $\ell_{it}$  increases to 2.44 the firm approaches the Basel II threshold. 

\subsection{Future returns}

Consider  firm $i$ at time  $t+h$ where $h>0$.  Then
$$
\ell_{i,t+h} = \ln \frac{d_{i,t+h}}{w_{i,t+h}} = \ell_{it} -\nu_{t}\cq \nu_{it}\equiv \ln\frac{w_{i,t+h}}{w_{it}}  \ ,
$$
where it is assumed $d_{i,t+h}=d_{it}$.  Hence $\nu_{it}$ is the return on equity over $(t,t+h)$. 
In terms of the Basel leverage $\ell_{it}^*$ at time $t$, the  Basel II shortfall  at time $t+h$ is,
\be{bs}
d_{it}k(1-\e^{\nu^*_{it}})\cq \nu_{it}^*\equiv \nu_{it}-\ell^*_{it}\ , 
\ee 
The Basel II limit is reached at $t+h$ if $\nu_{it}=\ell_{it}^*$ and  breached  if
$
\nu_{it}< \ell_{it}^*
$.
A breach is  unlikely if $\ell^*_{it}$ is low.
\newcommand{\Es}{\tilde\E}

Engle defines the systemic risk (SRISK) of a firm $i$ as the expected Basel II shortfall at $t+h$,
\be{esrisk}
 d_{it}\Es\{k(1-\e^{\nu^*_{it}})\} = d_{it}k\{1-\Es(\e^{\nu^*_{it}})\} \ ,
\ee
where the expectation $\Es$ is a ``stressed" expectation i.e. an expectation conditional on a stress in the system such as a major market downturn.




\subsection{Engle systemic risk}
The total SRISK of the system is defined as
\be{essrisk}
d_{it}k\{1-\Es(\e^{\nu^*_{it}})\} \cq  k\sum_i d_{it}|1-\Es(\e^{\nu^*_{it}})|^+\cq |x|^+\equiv \max(0,x) \ ,
\ee
respectively.


The following comments appear important:
\begin{enumerate}
\item The SRISK contribution of a firm is zero whenever
\be{cond}
\E(\e^{\nu_{it}})\le\e^{\ell^*_{it}}\ .
\ee
This inequality generally holds even if the expectation is severely stressed. 
\item  If $\nu_{it}$ is normal then 
$$
\E(\e^{\nu_{it}}) = \e^{h\mu_{it} +h^2\sigma^2_{it}/2}
$$
where the mean and standard deviation are the stressed versions.  In terms of this notation \eref{cond}
is equivalent to
$$
\mu_{it} +h\frac{\sigma^2_{it}}{2}\le \ell_{it}^* \ ,
$$
indicating a Basel breach is unlikely unless $\ell_{it}$ is near 2.44, the Basel limit.

\item The system SRISK is zero if all firm the SRISK contributions are zero that is if \eref{cond} holds for every $i$.  This would be usual in the Australian case.
\end{enumerate}

\subsection{Improved systemic risk measurement}

For each firm $i$ define the  Basel put and  Basel risk as 
\be{put}
p_{it}\equiv k |1-\e^{\nu^*_{it}}|^+\cq \E(p_{it})>0\ ,
\ee
where $\E$ denotes normal or unstressed expectation.  The Basel risk in firm $i$ depends only $k$,  the present leverage  and the lower tail of the  $\nu_{it}$ distribution.  The put value \eref{put} is stated in terms of unit debt and increases with $k$, both directly from  multiplication by $k$ and indirectly through its logit, making a positive put outcome more likely.  

Default put options similar to \eref{put} have been discussed in the insurance literature as critical to an evaluation of a firm:  see for
example \citet{merton1977analytic}, \citet{doherty1986price}, \citet{cummins1988risk}, \citet{myers2001capital} and \citet{sherris2006solvency}.

The total Basel put value and hence Basel risk for firm  $i$ is   $d_{it}\E(p_{it})$.   This is the cost of insuring firm $i$ does not breach the Basel standard.  In practice it may be appropriate to discount $p_{it}$ in \eref{put} by the interest rate over the period $(t,t+h)$ and value the put with risk neutral rather than normal expectation.       

The put $p_{it}$ is tradable since, given $k$, it relies only on the present leverage and the future return $\nu_{it}$ on equity $w_{it}$.  Market participants can value the put in  the same way as any other put and use the contract to diversify risk.  Firms can buy puts in the market to hedge Basel default risk.    Regulators will value puts using a stressed, rather than risk neutral, expectation and regulators can monitor stressed put values for systemic risk. Regulators can charge firms their put value if there are implicit government bailout guarantees.

In terms of the put values $\E(p_{it})$, the  system Basel risk is defined as the debt weighted average of the individual firm's put values:
\be{nsrisk}
\Ex\{\E(p_{it})\} = \E\{\Ex(p_{it})\}  \ . 
\ee
The standardised system Basel risk is stated per unit of system debt.  Further
$$
d_t\Ex\{\E(p_{it})\}  = \sum_id_{it}\E(p_{it}) \cq d_t\equiv\sum_id_{it}\ ,
$$
shows, on the left, the system Basel risk and, on the right, the weighted sum of firm Basel risks.  

\subsection{Stressed expectations}

In the above development $\E$  denotes  normal expectation.   A stressed expectation, denoted $\Es$, is 
linear combination of normal expectations:
\be{formula}
\Es(p_{it}) \equiv \E\{\phi\E(p_{it}|\phi)\} = \E(\phi p_{it}) = \E(p_{it}) + \cov(\phi,p_{it})\ .
\ee
Here $\phi\ge 0$ with $\E(\phi)=1$ is thought of as a ``stressor,"  an event or collection of events impacting put values and serving as a basis for stress testing.  In practice the scenarios $\phi$  downplay or highlight different scenarios.  The conditioning events are scenarios of interest and  weighting is according to the level of interest.


Writing  $\sigma_{it}$ as the unstressed standard deviation of $p_{it}$,  define the Basel stress of firm $i$ with respect to stress $\phi$ as
\be{zscore}
\psi_{it}\equiv \frac{\Es(p_{it})- \E(p_{it})}{\sigma_{it}}= \frac{\E\{(\phi-1)p_{it}\}}{\sigma_{it}} = \sigma_\phi c_{it} \ ,
\ee
where  $\sigma_\phi$ is the standard deviation of $\phi$ and $c_{it}$ the correlation of $\phi$ with $p_{it}$.  The measure $\psi_{it}$ captures, in units of put volatility, the increase in put values when  stress  $\phi$ is applied.  The constant $\sigma_\phi$ is analytically determined from $\phi$, while the correlation $c_{it}$  is  readily estimated using simulation.  (Interpretation of $\sigma_\phi c_{it}$ as a beta or z--score)

Stress events implicit in \eref{zscore} and captured with $\phi$ can be defined with respect any events.  Stresses are often defined with respect to systemic events such as a severe market downturn i.e. a large negative value for the forward return  $\nu_{mt}$.      In this context   the aim is to compute $\Es(p_{it})$  given stress in $\nu_{it}$ defined as an extreme   percentile $u_{mt}$ of $\nu_{mt}$.

The monetary Basel stress of firm $i$ at time $t$ is 
$$
d_{it}\sigma_{it}\psi_{it}\ .
$$

\subsection{System stress}

Systemic events are expected to affect all firms $i$.   A measure of the total effect is to consider the debt weighted put
$$
p_{t}\equiv \Ex(p_{it}) \cq \Es(p_{t}) -\E(p_{t}) = \cov(\phi,p_{t})\ .
$$
The ``system" put $p_t$ pays out zero if all firms meet the Basel requirement with payments increasing with the number of Basel breaches, the sizes of the breaches, and the relative debt sizes of the breaching firms. 

Define the system systemic risk as
\be{ssrisk}
\psi_t \equiv \frac{\Es(p_t)-\E(p_t)}{\sigma_t}=\frac{\Ex(\sigma_{it}\psi_{it})}{\sigma_t}\ ,
\ee
where  $\sigma_{t}$ is the (unstressed) standard deviation of the system put $p_{t}$.  The final equality   follows since the numerator in the middle expression equals
$$
 \cov\{\phi,\Ex(p_{it})\} =\Ex\{\cov(\phi,p_{it})\}=\Ex\{\Es(p_{it})-\E(p_{it})\}=\Ex(\sigma_{it}\psi_{it})\ .
$$
Thus $\psi_t$ is the response, in units of system put volatility $\sigma_t$, from stress.   

The monetary Basel system stress at time $t$ is the sum of monetary firm stresses
$$
d_t\sigma_t\psi_t = d_t\Ex(\sigma_{it}\psi_{it}) = \sum_i d_{it}\sigma_{it}\psi_{it}\ . 
$$

\subsection{Stressors}

If $\phi=\phi(u_{mt})$ with $\phi$  decreasing on $[0,1]$ then the stress is related to the the market return and there is larger stress if $u_{mt}$ is smaller. Hence the impact on $\E(p_{it})$ is larger if $\nu_{it}$ and $\nu_{mt}$ are highly correlated and 
$$
\tilde\E(p_{it}) = -\int_0^1 u\E(p_{it}|u_{mt}<u) \de \phi(u) \ ,
$$




The setup is illustrated with two examples which generalise the approach of Engle.

\subsubsection{Market return below a percentile threshold} 

If  $\phi(u)=1/\alpha$ for $u< \alpha$ and 0 otherwise then the stress event is a market return in the bottom $\alpha$--tail. Then
$$
\tilde\E(p_{it}) = \alpha\E\{I(u_{mt}<\alpha)p_{it}\} \approx \frac{1}{S/\alpha} \sum_s  I\left(u^s_{mt}<\frac{1}{\alpha}\right)p_{it}^s \ ,
$$
where $s=1,\ldots,S$ denotes simulations and $I$ the indicator function.  The approximation decreases with the simulation effort $S$.   The effective simulation effort is $S/\alpha$ and hence $\alpha$ small will require a very large $S$.

A fixed percentage threshold means the actual threshold depends on the distribution and hence  on volatility and  time.

\subsubsection{Expected worst market return in copies} 

Assume $\phi(u)=n(1-u)^{n-1}$ where $u$ is the market return percentile. Then the stressed expectation $\E$ is computed assuming the worst $h$ day return in $n$  independent and identical copies of the current situation, and
$$
\tilde\E(p_{it}) = n\E\{(1-u)^{n-1}p_{it}\} \approx \frac{1}{S/n} \sum_s (1-u^s_{mt})^{n-1}p_{it}^s \ . 
$$
Simulated returns $\nu^s_{mt}$ in the upper tail of the distribution  have percentiles $u^s_{mt}\approx 1$ and hence for these returns the second factor in the sum is  small if not negligable: the paired simulated put  is heavily downweighed.

Note the contrast with the previous example where the bottom $\alpha$ proportion of simulated market returns are selected as the stressed sample. In the latter situation every simulated value contributes to the expectation, albeit with vastly different weights.


\subsubsection{Worst market return given a tail event}

The above two situations can be combined.   Suppose  $\phi(u)=c(0.05-u)^{19}$ for $u\le 0.05$ and 0 otherwise and where $c$ is such that $\phi(u)$ integrates to 1.  Then
$$
\E(p_{it}) \approx \frac{1}{S/c}\sum_{u^s_{mt}<1/20} p_{it}^s \left(u^s_{mt}-\frac{1}{20}\right)^{20}\ .
$$
 Similar to the first example, the stressed sample picks up the bottom 5\% of market returns. The bottom 5\% of market returns is further stressed by  progressively downweighing returns as the percentile approaches 1/20.   Enormous simulation effort $S$ is required for a reasonable approximation since $S/c$ is the effective simulation size.

\subsection{Comparing SRISK for different firms (not sure)}

Suppose of interest is whether SRISK varies similarly or different across firms as the threshold $a$ varies. High correlation implies high systemic risk since firms are simultaneously affected in a market downturn. The correlation is
$$
\cor_a\left( \mathrm{SRISK}_i(a), \mathrm{SRISK}_2(a)\right) = \cor_a \left\{\E(r_1|u_m < a),\E(r_2|u_m < a)   \right\}
$$
where the correlation is computed using a uniform distribution for $c$. 


\begin{comment}
\subsection{Weihao - Allocating the market put}

The overall market shortfall after allowing for diversification between firms is
$$
p_{mt}\equiv k_+ |1-\e^{-\ell^*_{mt}+\nu_{mt}}|^+
= k_+|\Ex (1-\e^{-\ell^*_{it}+\nu_{it}})|^+
=I(\nu_{mt}<\ell_{mt}^*)k_+\Ex(1-\e^{-\ell^*_{it}+\nu_{it}}) 
$$
and hence the portion attributable to firm $i$ is $k_i(1-\e^{-\ell^*_{it}+\nu_{it}}) I(\nu_{mt}<\ell_{mt}^*)$, its capital shortfall or surplus when the overall market is in a shortfall. The allocation of the stressed expectation $\E(p_{mt})$ is then
$$
k_i\E\{(1-\e^{-\ell^*_{it}+\nu_{it}}) I(\nu_{mt}<\ell_{mt}^*)\}
$$
and applying $\Ex$ to the above expression yields $\E(p_{mt})$.
\end{comment}





\subsection{Contagion effects}
Definition \eref{nsrisk} can be generalized to capture contagion effects.  Consider 
\be{cont}
\Ex^j\{\E^j(p_{it})\}=\E^j\{\Ex^j(p_{it})\}\ ,
\ee
where  $\E^j$ denotes the stressed conditional expectation, conditioning on a firm $j$ breach, and $\Ex^j$ is an debt weighted average excluding firm $j$.  Firm $j$ is systemically important  if a Basel breach in firm $j$ leads to large increase in system systemic risk
$$
c_{ij}\equiv \frac{\E^j\{\Ex^j(p_{it})\}}{\E\{\Ex(p_{it})\}}-1\ .
$$

\subsection{Diversification effects} 

Suppose debt and equity are  aggregated across firms to yield $d_t$ and $w_t$.  The system log leverage is  
$$
\ell_t =  \ln \frac{d_t}{w_t} =  \ln \sum_i\frac{w_{it}}{w_t}\frac{ d_{it}}{ w_{it}} = \ln \Ex_w(\e^{\ell_{it}}) \ ,
$$
where $\Ex_w$ denotes an equity weighted average.  A system wide breach of the Basel II limit occurs at $t+h$ if
$
\nu_t < \ell_t^*\equiv \ell_t - 2.44
$
where $\nu_t$ is the rate of return on total equity $w_t$:
\be{ew}
 \nu_t = \ln \frac{w_{+t+h}}{w_t} = \ln \Ex_w\left(\frac{w_{i,t+h}}{w_{it}}\right) = \ln \Ex_w(\e^{\nu_{it}})\ . 
\ee

Similar to before define a put for the system and its (stressed) expected value
$$
p_t\equiv k|1-\e^{\nu_t-\ell_t^*}|^+\cq \E(p_t)\le\Ex\{\E(p_{it})\}\ .
$$
The inequality is a result  of diversification effects:   low liquidity  in one firm is implicitly offset by high liquidity  in other  firms.

\subsection{Risk weighted assets}

Suppose total assets of a company are $d_t+w_t$.   Total assets are often replaced by  risk weighted assets denoted $d_{it}+\e^{-v_{it}}w_{it}$ where $v_{it}>0$ is the implicit discounting on equity used to arrive at risk weighted assets.  
  Using risk weighted assets implies the modified leverage
$$
 \ln\left\{ \frac{d_{it}}{\e^{-v_{it}}w_{it}}\right\}=\ell_{it} +v_{it}\ .
$$
In terms of this setup the systemic risk in firm $i$ is then $\E(p_{it})$ computed with 
$$
p_{t}\equiv  k|1-\e^{\nu_{t}-\ell^*_{t}-v_{it}}|^+
$$
Since $v_{it}\ge 0$ the put is more valuable.   Note 
$$
v_{it} =-\ln\left\{ \frac{\Ex_\pi(a_{it})}{w_{it}} - \e^{\ell_{it}}\right\}
$$

\section*{References}
\bibliography{/users/pietdejong/documents/research/cifr/piet2}

\end{document}
